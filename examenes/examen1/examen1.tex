
\documentclass[letterpaper,11pt]{article}

% Soporte para los acentos.
\usepackage[utf8]{inputenc}
\usepackage[T1]{fontenc}    
% Idioma español.
\usepackage[spanish,mexico, es-tabla]{babel}
% Soporte de símbolos adicionales (matemáticas)
\usepackage{multirow}
\usepackage{amsmath}
\usepackage{amssymb}
\usepackage{amsthm}
\usepackage{amsfonts}
\usepackage{mathtools}
\usepackage{latexsym}
\usepackage{enumerate}
\usepackage{ragged2e}
\usepackage{listings}
\usepackage{xcolor}
\usepackage{graphicx}
\usepackage{hyperref}
\usepackage{drawstack}
% Modificamos los márgenes del documento.
\usepackage[lmargin=1cm,rmargin=1cm,top=1.5cm,bottom=1.5cm]{geometry}

\definecolor{codegreen}{rgb}{0,0.6,0}
\definecolor{codegray}{rgb}{0.5,0.5,0.5}
\definecolor{codepurple}{rgb}{0.58,0,0.82}
\definecolor{backcolour}{rgb}{0.95,0.95,0.92}

\lstdefinestyle{mystyle}{
    backgroundcolor=\color{backcolour},   
    commentstyle=\color{codegreen},
    keywordstyle=\color{magenta},
    numberstyle=\tiny\color{codegray},
    stringstyle=\color{codepurple},
    basicstyle=\ttfamily\footnotesize,
    breakatwhitespace=false,         
    breaklines=true,                 
    captionpos=b,                    
    keepspaces=true,                 
    numbers=left,                    
    numbersep=5pt,                  
    showspaces=false,                
    showstringspaces=false,
    showtabs=false,                  
    tabsize=2
}

\lstset{style=mystyle}

\tikzstyle{freecell}=[fill=blue!10,draw=blue!30!black]
\tikzstyle{occupiedcell}=[fill=blue!10!orange!10,draw=blue!30!black]
\tikzstyle{padding}=[fill=yellow!20,draw=blue!30!black]
\tikzstyle{highlight}=[draw=orange!50!black,text=orange!50!black]

\title{Facultad de Ciencias, UNAM \\ 
       Lenguajes de Programación \\ 
       Examen Parcial I}
\author{Rubí Rojas Tania Michelle }
\date{30 de octubre de 2020}

\begin{document}
\maketitle

\begin{enumerate}
    % Ejercicio 1.
    \item ¿Cuál es la complejidad del algoritmo de sustitución? Elabora tu 
    respuesta usando algún ejemplo.

    \textsc{Solución:} Sabemos que un programa tiene $n$ variables. El
    algoritmo hace la sustitución de las variables una por una, y por cada
    variable \textbf{de-ligado} hace el recorrido sobre las $m$ posibles
    variables más que haya sobre el resto del programa. Notemos también que 
    el algoritmo hace el recorrido por cada variable \textbf{de-ligado} (exista 
    o no dentro del programa). Entonces, en el peor caso, si tenemos $n$ 
    variables \textbf{de-ligado}, el algoritmo de sustitución se realizará en 
    cada una de las $n$ variables del programa; por lo que la complejidad del 
    algoritmo es de $O(n^2)$ (pues el recorrido que hace una variable 
    \textbf{de-ligado} en el programa nos toma $O(n)$, pero esto se realiza $n$
    veces).

    Un ejemplo que muestra esto es:
    \begin{verbatim}
    {with {x_1 4} 
     {with {x_2 4} 
       {with {x_3 4} 
         ...
         {with {x_n 4} 
           {+ x_1 {+ x_2 {+ x_3 ... {+ x_n-1 x_n}}}}}}}}
    \end{verbatim}

    donde tenemos $n$ variables \textbf{de-ligado} y $n$ variables 
    \textbf{ligadas}. En este ejemplo, a cada variable \textbf{de-ligado} le 
    corresponde una única variable \textbf{ligada}; pero el algoritmo no puede 
    saber ésto, así que por cada variable \textbf{de-ligado} realizará el 
    recorrido sobre las $n$ variables \textbf{ligadas} que contiene nuestro 
    programa para intentar sustituir el valor. Por lo que la complejidad será 
    de $O(n^2)$. 

    % Ejercicio 2.
    \item Da una expresión usando la gramática FWAE tal que una misma variable
    (supongamos $x$) aparezca en la misma expresión como instancia de ligado una 
    vez (con el mismo nombre de identificador), aparezca ligada al menos una 
    vez y aparezca exactamente una única vez como identificador libre.

    \textsc{Solución:} La expresión propuesta es
    \begin{equation*}
        \texttt{\textbf{\{with \{\textcolor{red}{x} \textcolor{blue}{x}\} 
                               \{+ \textcolor{green}{x} 
                                   \textcolor{green}{x}\}\}}}
    \end{equation*}

    Donde tenemos que la variable de color \textcolor{red}{rojo} es una variable 
    \textbf{de-ligado}, la variable de color \textcolor{blue}{azul} es una
    variable \textbf{libre} (pues tiene que salir a buscar su valor fuera de 
    esta expresión, y fuera de ésta no está ligada a nadie) y las variables de 
    color \textcolor{green}{verde} son  
    variables \textbf{ligadas}. 

    Otra expresión podría ser 
    \begin{equation*}
        \texttt{\textbf{\{+ \textcolor{blue}{x} 
                            \{with \{\textcolor{red}{x} 2\} 
                                   \{\textcolor{green}{x} 
                                     \textcolor{green}{x}\}\}\}}}
    \end{equation*}

    donde la variable de color \textcolor{blue}{azul} es una variable 
    \textbf{libre} (pues no está ligada a nadie), la variable de color 
    \textcolor{red}{rojo} es una variable \textbf{de-ligado} y las 
    variables de color \textcolor{green}{verde} son \textbf{ligadas}.

    \newpage
    % Ejercicio 3.
    \item Convierte el siguiente código usando índices de Bruijn o direcciones
    léxicas. 
    \begin{verbatim}
    {with {a -1} 
      {with {b 1} 
        {with {c 2} 
          {with {d {+ 2 1}} 
            {* d {/ c {+ {* b a} {+ a a}}}}}}}}
    \end{verbatim}

    \textsc{Solución:}
    \begin{verbatim}
    {with -1 
      {with 1 
        {with 2 
          {with {+ 2 1} 
            {* <:0> {/ <:1> {+ {* <:2> <:3>} {+ <:3> <:3>}}}}}}}}
    \end{verbatim}

    % Ejercicio 4.
    \item Convierte el siguiente código con índices de Bruijn a código dentro 
    de la gramática WAE. Las instancias de ligado se llaman $x, y, z, a, b$ con 
    respecto al órden de aparición de las mismas. 
    \begin{verbatim}
    {with 1 
      {with 2
        {with 3
          {with {* <:0 0> <:1 0>} 
            {with 5
              {+ <:0 0> {+ <:1 0> {+ <:2 0> {+ <:3 0> <:4 0>}}}}}}}}}
    \end{verbatim}

    \textsc{Solución:}
    \begin{verbatim}
    {with {x 1} 
      {with {y 2} 
        {with {z 3} 
          {with {a {* z y}} 
            {with {b 5} 
              {+ b {+ a {+ z {+ y x}}}}}}}}}
    \end{verbatim}

    % Ejercicio 5.
    \item ¿A qué se le conoce como azúcar sintáctica en un lenguaje de 
    programación?

    \textsc{Solución:} La azúcar sintáctica es una especie de \textit{sintáxis 
    especial} que hace más fácil la escritura de algunas expresiones en un  
    lenguaje de programación, haciéndo que éstas sean más \textit{dulces} para 
    el usuario. Por ejemplo, \texttt{with} es azúcar sintáctica de una 
    aplicación de función.

    % Ejercicio 6.
    \item Ponga el ambiente en forma de pila (stack) para la siguiente 
    expresión, y evalúe la siguiente expresión usando 
    \begin{enumerate}
        \item Alcance estático
        \item Alcance dinámico
    \end{enumerate}

    es necesario especificar cada una de las expresiones a evaluar con los 
    respectivos valores. 
    \begin{verbatim}
    {with {a 1} 
      {with {b 1} 
        {with {a 0} 
          {with {foo1 {fun {x} {* x {+ b a}}}} 
             {with {b 0} 
                {with {a 1} 
                   {foo1 2}}}}}}}
    \end{verbatim}

    \textsc{Solución:}
    \begin{enumerate}
        \item Alcance estático

        La expresión que debemos evaluar es \texttt{\{foo1 2\}}, por lo que 
        \begin{verbatim}
        {foo1 2} = {{fun {x} {* x {+ b a}}} 2}
                 = {* x {+ b a}}
                 = {* 2 {+ 1 0}}
                 = {* 2 1}
                 = 2
        \end{verbatim}

        Además, el ambiente en forma de pila correspondiente es:
        \begin{center}
            \begin{drawstack}[scale=1.6]
                \cell{\texttt{a} \quad \quad $1$}
                \cell{\texttt{b} \quad \quad $0$} 
                \cell{\texttt{foo1 \quad \{fun \{x\} \{* x \{+ b a\}\}\}}}
                \bcell{\texttt{x} \quad \quad $2$}
                \cell{\texttt{a} \quad \quad $0$}
                \cell{\texttt{b} \quad \quad $1$}
                \cell{\texttt{a} \quad \quad $1$}
            \end{drawstack}
        \end{center}

        \item Alcance dinámico

        La expresión que debemos evaluar es \texttt{\{foo 2\}}, por lo que
        \begin{verbatim}
        {foo 2} = {{fun {x} {* x {+ b a}}} 2}
                = {* x {+ b a}}
                = {* 2 {+ 0 1}}
                = {* 2 1}
                = 2
        \end{verbatim}

        Además, el ambiente en forma de pila correspondiente es:
        \begin{center}
            \begin{drawstack}[scale=1.6]
                \bcell{\texttt{x} \quad \quad $2$}
                \cell{\texttt{a} \quad \quad $1$}
                \cell{\texttt{b} \quad \quad $0$} 
                \cell{\texttt{foo1 \quad \{fun \{x\} \{* x \{+ b a\}\}\}}}
                \cell{\texttt{a} \quad \quad $0$}
                \cell{\texttt{b} \quad \quad $1$}
                \cell{\texttt{a} \quad \quad $1$}
            \end{drawstack}
        \end{center}
    \end{enumerate}

    \newpage
    % Ejercicio 7.
    \item Escribe la definición de la función recursiva \texttt{agregaN} en 
    \textit{Racket}, que reciba tres parámetros: un elemento $e$, un número 
    $n$ y una lista $l$. La función debe regresar la lista resultante de 
    agregar a la lista $l$, el elemento $e$ en la posición $n$. No puedes 
    utilizar ninguna función nativa del lenguaje, exceptuándo los operadores
    aritméticos \texttt{car}, \texttt{cons}, \texttt{append} e \texttt{equal?}.
    \begin{verbatim}
    ;; agregaN: a exact list -> list 

    (agregaN 2 7 (list 1 2 3 4 5 6 7 8))
    > '(1 2 3 4 5 6 7 2 8)
    \end{verbatim}

    \textsc{Solución:}
    \begin{lstlisting}[language=Lisp]
        ;; agregaN: a exact list -> list
        (define (agregaN e n l)
          (cond
            [(< n 0) error 'agregaN "El indice debe ser un entero positivo."]
            [(= n 0) (cons e l)]
            [else (cons (car l) (agregaN e (- n 1) (cdr l)))]))
    \end{lstlisting}
\end{enumerate}

\end{document}
