\documentclass[letterpaper,11pt]{article}

% Soporte para los acentos.
\usepackage[utf8]{inputenc}
\usepackage[T1]{fontenc}
% Idioma español.
\usepackage[spanish,mexico, es-tabla]{babel}
% Soporte de símbolos adicionales (matemáticas)
\usepackage{amsmath}
\usepackage[dvipsnames]{xcolor}
% Modificamos los márgenes del documento.                                       
\usepackage[lmargin=1.5cm,rmargin=1.5cm,top=1.5cm,bottom=1.5cm]{geometry}

\title{Facultad de Ciencias, UNAM \\ 
       Lenguajes de Programación\\ 
       Examen Parcial 2}
\author{Rubí Rojas Tania Michelle}
\date{07 de enero de 2021}

\begin{document}
\maketitle

\begin{enumerate}
    % Pregunta 1.
    \item Evalúa la siguiente expresión usando representación de ambientes en 
    todos los casos. Es necesario expresar el ambiente final (stack) para 
    evaluación glotona y perezosa; además de la expresión completa a evaluar 
    en cada uno de los incisos anteriores, antes de dar el resultado final de 
    tales evaluaciones.
    \begin{verbatim}
    {with {x {+ 2 2}}
      {with {y {+ 1 2}}
        {with {z 3}}
          {with {foo {fun {x} {+ x {+ y z}}}}
            {with {x 3}
              {with {y {+ 2 2}}
                {with {z {+ 1 1}}
                  {with {mas-foo {fun {y} {* 1 {* x y}}}}
                    {with {x 2}
                      {mas-foo 3}}}}}}}}}
    \end{verbatim}

    \begin{itemize}
        % Ejercicio 1.a
        \item Evaluación perezosa y alcance estático.

        \textsc{Solución:} La expresión que debemos evaluar es 
        \texttt{\{mas-foo 3\}}, por lo que 
        \begin{table}[h]
            \parbox{.54\linewidth}{
            \centering
            \begin{align*}
                \texttt{\{mas-foo 3\}} 
                &= \texttt{\{\{fun \{y\} \{* 1 \{* x y\}\}\} 3\}} \\
                &= \texttt{\{* 1 \{* x 3\}\}} \\
                &= \texttt{\{* 1 \{* 3 3\}\}} \\ 
                &= \texttt{\{* 1 9\}} \\
                &= 9 
            \end{align*}
            }
            \hfill
            \parbox{.45\linewidth}{
            \centering
            \begin{tabular}{|c|c|}
            \hline
            \texttt{x} & \texttt{2} \\ 
            \hline
            \texttt{mas-foo} & \texttt{[closureV: y, \{* 1 \{* x y\}\},} \\
            & \texttt{env-ant: (y 3), env7]} \\
            \hline
            \texttt{z} & \texttt{\{+ 1 1\}} \\ 
            \hline
            \texttt{y} & \texttt{\{+ 2 2\}} \\
            \hline
            \texttt{x} & \texttt{3} \\
            \hline
            \texttt{foo} & \texttt{[closureV: x,\{+ x \{+ y z\}\},} \\
            & \texttt{env-ant:env3]} \\
            \hline
            \texttt{z} & \texttt{3} \\
            \hline
            \texttt{y} & \texttt{\{+ 1 2\}} \\
            \hline
            \texttt{x} & \texttt{\{+ 2 2\}} \\
            \hline
            \end{tabular}
            \caption{Ambiente final}
            }
        \end{table}

        \newpage
        % Ejercicio 1.b
        \item Evaluación perezosa y alcance dinámico.

        \textsc{Solución:} La expresión que debemos evaluar es 
        \texttt{\{mas-foo 3\}}, por lo que 
        \begin{table}[h]
            \parbox{.54\linewidth}{
            \centering
            \begin{align*}
                \texttt{\{mas-foo 3\}} 
                &= \texttt{\{\{fun \{y\} \{* 1 \{* x y\}\}\} 3\}} \\
                &= \texttt{\{* 1 \{* x 3\}\}} \\
                &= \texttt{\{* 1 \{* 2 3\}\}} \\ 
                &= \texttt{\{* 1 6\}} \\
                &= 6
            \end{align*}
            }
            \hfill
            \parbox{.45\linewidth}{
            \centering
            \begin{tabular}{|c|c|}
            \hline
            \texttt{x} & \texttt{2} \\ 
            \hline
            \texttt{mas-foo} & \texttt{\{fun \{y\} \{* 1 \{* x y\}\}\}} \\
            \hline
            \texttt{z} & \texttt{\{+ 1 1\}} \\ 
            \hline
            \texttt{y} & \texttt{\{+ 2 2\}} \\
            \hline
            \texttt{x} & \texttt{3} \\
            \hline
            \texttt{foo} & \texttt{\{fun \{x\} \{+ x \{+ y z\}\}\}} \\
            \hline
            \texttt{z} & \texttt{3} \\
            \hline
            \texttt{y} & \texttt{\{+ 1 2\}} \\
            \hline
            \texttt{x} & \texttt{\{+ 2 2\}} \\
            \hline
            \end{tabular}
            \caption{Ambiente final}
            }
        \end{table}

        % Ejercicio 1.c
        \item Evaluación glotona y alcance estático.
        
        \textsc{Solución:} La expresión que debemos evaluar es 
        \texttt{\{mas-foo 3\}}, por lo que 
        \begin{table}[h]
            \parbox{.54\linewidth}{
            \centering
            \begin{align*}
                \texttt{\{mas-foo 3\}} 
                &= \texttt{\{\{fun \{y\} \{* 1 \{* x y\}\}\} 3\}} \\
                &= \texttt{\{* 1 \{* x 3\}\}} \\
                &= \texttt{\{* 1 \{* 3 3\}\}} \\ 
                &= \texttt{\{* 1 9\}} \\
                &= 9 
            \end{align*}
            }
            \hfill
            \parbox{.45\linewidth}{
            \centering
            \begin{tabular}{|c|c|}
            \hline
            \texttt{x} & \texttt{2} \\ 
            \hline
            \texttt{mas-foo} & \texttt{[closureV: y, \{* 1 \{* x y\}\},} \\
            & \texttt{env-ant: ((y 3), env7)]} \\
            \hline
            \texttt{z} & \texttt{2} \\ 
            \hline
            \texttt{y} & \texttt{4} \\
            \hline
            \texttt{x} & \texttt{3} \\
            \hline
            \texttt{foo} & \texttt{[closureV: x,\{+ x \{+ y z\}\},} \\
            & \texttt{env-ant: env3]} \\
            \hline
            \texttt{z} & \texttt{3} \\
            \hline
            \texttt{y} & \texttt{3} \\
            \hline
            \texttt{x} & \texttt{4} \\
            \hline
            \end{tabular}
            \caption{Ambiente final}
            }
        \end{table}

        % Ejercicio 1.d
        \item Evaluación glotona y alcance dinámico.

        \textsc{Solución:} La expresión que debemos evaluar es 
        \texttt{\{mas-foo 3\}}, por lo que 
        \begin{table}[h]
            \parbox{.54\linewidth}{
            \centering
            \begin{align*}
                \texttt{\{mas-foo 3\}} 
                &= \texttt{\{\{fun \{y\} \{* 1 \{* x y\}\}\} 3\}} \\
                &= \texttt{\{* 1 \{* x 3\}\}} \\
                &= \texttt{\{* 1 \{* 2 3\}\}} \\ 
                &= \texttt{\{* 1 6\}} \\
                &= 6
            \end{align*}
            }
            \hfill
            \parbox{.45\linewidth}{
            \centering
            \begin{tabular}{|c|c|}
            \hline
            \texttt{x} & \texttt{2} \\ 
            \hline
            \texttt{mas-foo} & \texttt{\{fun \{y\} \{* 1 \{* x y\}\}\}} \\
            \hline
            \texttt{z} & \texttt{2} \\ 
            \hline
            \texttt{y} & \texttt{4} \\
            \hline
            \texttt{x} & \texttt{3} \\
            \hline
            \texttt{foo} & \texttt{\{fun \{x\} \{+ x \{+ y z\}\}\}} \\
            \hline
            \texttt{z} & \texttt{3} \\
            \hline
            \texttt{y} & \texttt{3} \\
            \hline
            \texttt{x} & \texttt{4} \\
            \hline
            \end{tabular}
            \caption{Ambiente final}
            }
        \end{table}
    \end{itemize}

    \newpage
    % Ejercicio 2.
    \item Sea una función $f$ definida en el lenguaje de programación 
    \textsc{Racket} como \texttt{number -> number}. Explica con tus propias 
    palabras por qué si siempre:
    \begin{equation*}
        2 * (f \; x) == (f \; x) + (f \; x)
    \end{equation*}

    entonces es un ejemplo que muestra la transparencia referencial de $f$.

    \textsc{Solución:}

    % Ejercicio 3.
    \item Transforma la siguiente función usando recursión de cola, y agrega
    los registros de activación de la llamada a función de 
    \texttt{(division 4 2)}.
    \begin{verbatim}
    (define division
      (lambda (n m)
        (if (= n 0)
            1
            (+ 1 (division (- n m) m)))))
    \end{verbatim}

    \textsc{Solución:} La función \texttt{division} puede ser optimizada 
    usando recursión de cola de la siguiente manera 
    \begin{verbatim}
    (define division-tail
      (local ((define sos 
                (lambda (n m acc)
                  (if (= n 0)
                      acc
                      (sos (- n m) m (+ 1 acc))))))
        (lambda (n m)
          (sos n m 1))))

    (define division 
      division-tail)
    \end{verbatim}

    Por otro lado, los registros de activación de la llamada a función de 
    \texttt{(division 4 2)} son:

    Entra \texttt{(division 4 2)}
    \begin{center}
        \begin{tabular}{|c|}
            \hline
            \texttt{division-tail 4 2} \\
            \texttt{division-tail} \\
            \texttt{4 2} \\
            \texttt{division} \\
            \hline
        \end{tabular}
    \end{center}

    Entra \texttt{(division-tail 4 2)}
    \begin{center}
        \begin{tabular}[h]{|c|}
            \hline
            \texttt{(sos 4 2 1)} \\
            \texttt{(local ((define sos 
                              (lambda (n m acc)} \\
            \texttt{(if (= n 0)
                        acc} \\
            \texttt{(sos (- n m) m (+ 1 acc))))))} \\  
            \texttt{(lambda (n m)
                      (sos n m 1)))} \\
            \texttt{4 2} \\ 
            \texttt{division-tail} \\
            \hline
        \end{tabular}
    \end{center}

    Entra$/$Sale \texttt{(sos 4 2 1)}
    \begin{center}
        \begin{tabular}[h]{|c|}
            \hline
            \texttt{(sos 2 2 2)} \\
            \texttt{(if (= n 0)
                        acc} \\
            \texttt{(sos (- n m) m (+ 1 acc)))} \\ 
            \texttt{4 2 1} \\
            \texttt{sos} \\
            \hline
        \end{tabular}
    \end{center}

    Entra$/$Sale \texttt{(sos 2 2 2)}
    \begin{center}
        \begin{tabular}[h]{|c|}
            \hline
            \texttt{(sos 0 2 3)} \\
            \texttt{(if (= n 0)
                        acc} \\
            \texttt{(sos (- n m) m (+ 1 acc)))} \\ 
            \texttt{2 2 2} \\
            \texttt{sos} \\
            \hline
        \end{tabular}
    \end{center}

    Entra$/$Sale \texttt{(sos 0 2 3)}
    \begin{center}
        \begin{tabular}[h]{|c|}
            \hline
            \texttt{3} \\
            \texttt{(if (= n 0)
                        acc} \\
            \texttt{(sos (- n m) m (+ 1 acc)))} \\ 
            \texttt{0 2 3} \\
            \texttt{sos} \\
            \hline
        \end{tabular}
    \end{center}

    Donde finalmente obtenemos el valor de $3$.

    % Ejercicio 4.
    \item Dentro del Cálculo Lambda, evalúa cada una de las siguientes 
    expresiones usando $\beta -$reducciones. Si alguna tiene forma normal, 
    especifícala.
    \begin{itemize}
        % Ejercicio 4.a
        \item $(\lambda x.x)(\lambda x.xxx)$
        
        \textsc{Solución:}
        \begin{align*}
            (\lambda x.x)(\lambda x.xxx) 
            &\rightarrow_{\beta} x [x := (\lambda x.xxx)] \\ 
            &\rightarrow_{\beta} \lambda x.xxx
        \end{align*}

        Como la expresión $\lambda x.xxx$ ya no puede reducirse más mediante 
        $\beta -$reducciones, entonces ya se encuentra en Forma Normal.

        % Ejercicio 4.b
        \item $(\lambda x.(\lambda y.yxw) \; z) \; u$
        
        \textsc{Solución:}
        \begin{align*}
            (\lambda x.(\lambda y.yxw) \; z) \; u
            &\rightarrow_{\beta} (\lambda y.yxw[x := z]) \; u \\ 
            &\rightarrow_{\beta} (\lambda y.yzw) \; u \\ 
            &\rightarrow_{\beta} yzw[y := u] \\
            &\rightarrow_{\beta} uzw 
        \end{align*}

        Como la expresión $uzw$ ya no puede reducirse más mediante 
        $\beta -$reducciones, entonces ya se encuentra en Forma Normal.

        % Ejercicio 4.c
        \item $(\lambda x.\lambda y.\lambda z.x) (yz)$
        
        \textsc{Solución:}
        \begin{align*}
            (\lambda x.\lambda y.\lambda z.x) (yz)
            &\rightarrow_{\beta} 
            \lambda y.\lambda z.x[x := (yz)] \\
            &\rightarrow_{\beta}
            \lambda y.\lambda z.yz
        \end{align*}

        Como la expresión $\lambda y.\lambda z.yz$ ya no puede reducirse más 
        mediante $\beta -$reducciones, entonces ya se encuentra en Forma Normal.
    \end{itemize}

    % Ejercicio 5.
    \item Define el combinador $Y$ de manera formal. Da un ejemplo de uso de 
    éste.

    \textsc{Solución:} El combinador $Y$ se define formalmente como sigue 

    \begin{equation*}
        Y =_{def} \lambda f.(\lambda x.f (xx)) (\lambda x.f (xx))
    \end{equation*}

    % Ejercicio 6.
    \item Da un ejemplo en \textsc{Racket} donde uses la estructura de cajas y
    expongas el concepto de estado y SPS. En el ejemplo se debe reflejar el 
    cambio de estado de una variable definida como una caja, cuyo valor dentro 
    de la caja sea inicialmente $0$ (cero) y termine con valor de $2$, teniéndo 
    un valor de $1$ de forma intermedia.

    \textsc{Solución:} Definimos la expresión \texttt{expr} como 
    \begin{verbatim}
    '{with {b {newbox 0}}
       {seqn {setbox b {+ 1 {openbox b}}}
             {setbox b {+ 1 {openbox b}}}
             {openbox b}}}
    \end{verbatim}
    
    Para interpretar un programa, llamamos a la función \texttt{interp} 
    con un ambiente y un Store (los cuales vimos en clase).
    \begin{verbatim}
    (vxs-value (interp (parse expr) (mtSub) (mtSto)))
    \end{verbatim}

    Lo cual nos regresa como resultado \texttt{2}. Notemos que inicialmente
    la caja tiene un valor de $0$, luego adquiere un valor de $1$ al ejecutarse 
    la primer línea del \texttt{seqn} y finalmente toma un valor de $2$ al 
    ejecutarse la segunda líneal del \texttt{seqn}. Como la última línea de 
    éste es la que se regresa, entonces obtenemos el valor actual de la caja, 
    el cual es $2$.
\end{enumerate}

\end{document}
