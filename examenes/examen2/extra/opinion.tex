\documentclass[letterpaper,11pt]{article}

% Soporte para los acentos.
\usepackage[utf8]{inputenc}
\usepackage[T1]{fontenc}
% Idioma español.
\usepackage[spanish,mexico, es-tabla]{babel}
% Soporte de símbolos adicionales (matemáticas)
\usepackage{amsmath}
\usepackage[dvipsnames]{xcolor}
% Modificamos los márgenes del documento.                                       
\usepackage[lmargin=3cm,rmargin=3cm,top=3cm,bottom=3cm]{geometry}

\title{Facultad de Ciencias, UNAM \\ 
       Lenguajes de Programación\\ 
       Reflexión del prefacio \textit{"Elogio a la Pereza"}}
\author{Rubí Rojas Tania Michelle}
\date{07 de diciembre de 2020}

\begin{document}
\maketitle

Como bien escribe el buen profesor \texttt{Galaviz}: \textit{"\;una labor que 
cumple con todos los requisitos para dar pereza a cualquier persona saludable, 
es la de hacer cálculos aritméticos"}. Y no sólo le puede dar pereza a una 
persona, sino también a una computadora. Esto nos lo demuestra la evaluación 
perezosa, pues sólo evalúa los argumentos de una función hasta que es 
estrictamente necesario (es decir, en los puntos estrictos). ¿Será este tipo 
de evaluación obra del demonio \texttt{Belphegor}? Creo que podemos pensar 
en una teoría:
\begin{itemize}
    \item \texttt{Belphegor}, al enterarse de que su engañoso camino nos llevó 
    a crear una ciencia, se alegró tanto que sintió curiosidad por su creación.
    Puede que el área de \texttt{Lenguajes de programación} le gustara tanto, 
    que inspiró a alguna persona para que implementara la evaluación perezosa, 
    en honor a él. Probablemente haya dejado su marca personal en otras áreas 
    de la computación, y es un buen ejercicio de reflexión el tratar de 
    averiguar cuáles son.
\end{itemize}

Es interesante pensar como la flojera o la pereza la podemos ver en varios 
aspectos de la computación, hasta en los más mínimos detalles, como es la 
evaluación perezosa. Si bien la mayoría de los lenguajes de programación no 
implementan este tipo de evaluación, creo que trae consigo una ventaja muy 
bonita: \texttt{las listas infinitas}. Un lenguaje tan matemático como 
\textsc{Haskell} es capaz de sacarle el mayor beneficio a esta característica, 
pues es justo en matemáticas donde son utilizados conjuntos infinitos (los 
números naturales, enteros, reales, etc). Tal vez sea la belleza de que existen 
unos infinitos más grandes que otros lo que inspiró a \texttt{Belphegor} para 
dotar a la evaluación perezosa con esta característica. O quizá sea la belleza 
infinita del universo (o del lugar donde habita) lo que hizo que nos 
proporcionara tal cualidad. Creo que nunca lo sabremos, pero, mientras tanto, 
es buena idea aprovechar esta gran herramienta que tenemos para crear cosas 
muy bonitas utilizándo listas infinitas (sin perder el estilo de la pereza, 
claro está).

\end{document}
