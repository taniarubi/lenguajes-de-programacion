\documentclass[letterpaper,11pt]{article}

% Soporte para los acentos.
\usepackage[utf8]{inputenc}
\usepackage[T1]{fontenc}
% Idioma español.
\usepackage[spanish,mexico, es-tabla]{babel}
% Soporte de símbolos adicionales (matemáticas)
\usepackage{amsmath}
\usepackage[dvipsnames]{xcolor}
\usepackage{array}
\newcolumntype{L}{>{\centering\arraybackslash}m{6cm}}
% Modificamos los márgenes del documento.                                       
\usepackage[lmargin=1.5cm,rmargin=1.5cm,top=1.5cm,bottom=1.5cm]{geometry}

\title{Facultad de Ciencias, UNAM \\ 
       Lenguajes de Programación\\ 
       Características de la evaluación perezosa y glotona}
\author{Rubí Rojas Tania Michelle}
\date{07 de diciembre de 2020}

\begin{document}
\maketitle

% Ejemplo 1.
\begin{table}[ht]
    \centering  
    \begin{tabular}{|L|L|}
        \hline
        Evaluación perezosa & Evaluación glotona \\
        \hline 
        Los argumentos de una función no son evaluados hasta que es 
        estrictamente necesario. &
        Los argumentos de una función siempre deben ser reducidos a un valor.\\
        \hline
    \end{tabular}
\end{table}

\textsc{Ejemplo:} Definimos la función \texttt{suma} como
\begin{verbatim}
    (define (suma l)
      (if (empty? l)
          0
          (+ (car l) (suma (cdr l)))))
\end{verbatim}

La evaluación de esta función con el parámetro \texttt{'(30 26)}
utilizándo evaluación perezosa sería de la siguiente forma:
\begin{verbatim}
(suma '(30 26)) = (+ (car '(30 26)) (suma (cdr '(30 26))))
                = (+ (car '(30 26)) (suma '(26)))
                = (+ (car '(30 26)) 
                     (+ (car '(26)) (suma (cdr '(26)))))
                = (+ (car '(30 26)) 
                     (+ (car '(26)) (suma '())))
                = (+ (car '(30 26)) 
                     (+ (car '(26)) 0))
                = (+ (car '(30 26)) 
                        (+ 26 0))
                = (+ (car '(30 26)) 26)
                = (+ 30 26)
                = 56
\end{verbatim}

Mientras que la evaluación de la función \texttt{suma} con el parámetro 
\texttt{'(30 26)} utilizándo evaluación perezosa sería de la siguiente forma:
\begin{verbatim}
(suma '(30 26)) = (+ (car '(30 26)) (suma (cdr '(30 26))))
                = (+ 30 (suma '(26)))
                = (+ 30 (+ (car '(26)) (suma (cdr '(26)))))
                = (+ 30 (+ 26 (suma '())))
                = (+ 30 (+ 26 0))
                = (+ 30 26)
                = 56
\end{verbatim}

% Ejemplo 2.
\begin{table}[ht]
    \centering  
    \begin{tabular}{|L|L|}
        \hline
        Evaluación perezosa & Evaluación glotona \\
        \hline 
        Es más eficiente cuando hay variables que nunca usamos & 
        Es menos eficiente cuando hay variables que nunca 
        usamos. \\
        \hline
    \end{tabular}
\end{table}

\textsc{Ejemplo:} Definimos las funciones \texttt{fact} y \texttt{goo} como 
sigue 
\begin{verbatim}
    (define (fact n) 
      (if (zero? n)
          1
          (* n (fact (sub1 n)))))
          
    (define (goo n)
      (let ([x (fact 10000)])
        (let ([y (fact 20000)])
          (let ([z (fact 30000)])
            n))))
\end{verbatim}

Si llamamos a \texttt{(goo 100000)}, utilizándo evaluación perezosa, como 
nunca utilizamos ninguna aparición de \texttt{fact}, entonces en nuestra 
pila de ejecución no tendremos la evaluación de cada una de las expresiones 
\texttt{fact}. Por otro lado, utilizándo evaluación glotona tendremos que 
evaluar cada una de las apariciones de la función \texttt{fact}, lo que 
hará que esta evaluación sea más lenta y ocupe más espacio en memoria, en 
comparación con la evaluación perezosa, la cual será más rápida y ocupará 
menos espacio en memoria.

% Ejemplo 3.
\begin{table}[h]
    \centering  
    \begin{tabular}{|L|L|}
        \hline
        Evaluación perezosa & Evaluación glotona \\
        \hline 
        Puede que no detecte errores en semántica (en la expresión no 
        evaluada) & Siempre detecta errores de semántica \\
        \hline
    \end{tabular}
\end{table}

\textsc{Ejemplo:} Definimos la función \texttt{foo} como sigue
\begin{verbatim}
    (define (foo n)
      (let ([x (/ 7 0)])
        n))
\end{verbatim}

Notemos que la expresión \texttt{(/ 7 0)} ocasiona un error, pues no es posible 
dividir entre cero. Utilizándo evaluación perezosa, como la expresión 
\texttt{(/ 7 0)} no es usada, entonces esta asignación es ignorada (se realiza 
la asignación, pero no la aplicación de función), por lo que simplemente 
regresaremos el valor que le pasamos como parámetro a la función \texttt{foo}.
Por otro lado, utilizándo evaluación glotona, la $n$ ni siquiera llega a 
evaluarse, pues detectaría el error al intentar dividir entre cero.

% Ejemplo 4.
\begin{table}[h]
    \centering  
    \begin{tabular}{|L|L|}
        \hline
        Evaluación perezosa & Evaluación glotona \\
        \hline 
        En promedio, tiene peor uso de espacio & En promedio, tiene mejor uso
        de espacio \\
        \hline
    \end{tabular}
\end{table}

\textsc{Ejemplo:} Definimos la función \texttt{hoo} como sigue: 
\begin{verbatim}
    (define (hoo l)
      (if (empty? l)
          '()
          (cons (+ (car l) 1) (hoo (cdr l)))))
\end{verbatim}

\newpage
La evaluación de esta función con el argumento \texttt{'(5)} utilizándo 
evaluación perezosa sería de la forma:
\begin{verbatim}
    (hoo '(5)) = (cons (+ (car '(5)) 1) (hoo (cdr '(5))))
               = (cons (+ (car '(5)) 1) (hoo '()))
               = (cons (+ (car '(5)) 1) '())
               = (cons (+ 5 1) '())
               = (cons 6 '())
               = '(6)
\end{verbatim}

Mientras que la evaluación de la función con el mismo argumento utilizándo 
evaluación glotona sería de la forma:
\begin{verbatim}
    (hoo '(5)) = (cons (+ (car '(5)) 1) (hoo (cdr '(5))))
               = (cons (+ 5 1) (hoo '()))
               = (cons 6 '())
               = '(6)
\end{verbatim}

De esta forma, podemos notar que usando evaluación perezosa estamos cargando 
en memoria con las expresiones que no aún no evaluamos, mientras que la 
evaluación glotona no lo hace (pues evalúa las expresiones terminales 
inmediatamente). Por esta razón, la evaluación glotona tiene menor complejidad 
en espacio que la evaluación perezosa.

% Ejemplo 6: Evaluación perezosa
\begin{table}[h]
    \centering  
    \begin{tabular}{|L|L|}
        \hline
        Evaluación perezosa \\
        \hline 
        Es posible definir estructuras de datos infinitas. \\
        \hline
    \end{tabular}
\end{table}

\textsc{Ejemplo:} Como \textsc{Haskell} utiliza evaluación perezosa, entonces 
soporta listas infinitas. Si una función no tiene casos base, entonces podemos 
seguir calculando algo infinitamente (o bien, produciéndo una estructura 
infinita). Sin embargo, lo bueno de estas listas es que podemos 
\textit{cortarlas} por donde queramos. La función \texttt{repeat} toma un 
elemento y regresa una lista infinita que simplemente tiene ese elemento. 
Una implementación recursiva podría ser como sigue:
\begin{verbatim}
    repeat :: a -> [a]
    repeat x = x : repeat x
\end{verbatim}

Si llamamos a \texttt{repeat 5} entonces obtendríamos una lista que tiene un 
$5$ en su cabeza y luego tendría una lista infinita de cincos en su cola, es 
decir, una lista de la forma \texttt{[5 5 5 5 ...]} que nunca terminaría su 
evaluación. Pero, si usamos la función \texttt{take} entonces podríamos cortar 
la lista. De esta forma, al llamar \texttt{take 3 (repeat 5)} obtenemos una 
lista con tres cincos, es decir, una lista de la forma \texttt{[5 5 5]}.

% Ejemplo 7.
\begin{table}[h]
    \centering  
    \begin{tabular}{|L|L|}
        \hline
        Evaluación glotona \\
        \hline 
        Se pierde el contexto o referencia del origen de las variables. \\
        \hline
    \end{tabular}
\end{table}

\textsc{Ejemplo:} Recórdando el ejemplo número $1$ (usándo evaluación glotona) 
podemos notar cómo vamos perdiéndo las referencias acerca de dónde obtenemos 
cada valor. 
\begin{verbatim}
(suma '(30 26)) = (+ (car '(30 26)) (suma (cdr '(30 26))))
                = (+ 30 (suma '(26)))
                = (+ 30 (+ (car '(26)) (suma (cdr '(26)))))
                = (+ 30 (+ 26 (suma '())))
                = (+ 30 (+ 26 0))
                = (+ 30 26)
                = 56
\end{verbatim}
 
Por ejemplo, al final, el número $30$ no sabémos de dónde lo obtenemos, pues 
hemos perdido su contexto de origen.


\end{document}
