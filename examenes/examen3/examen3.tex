\documentclass[letterpaper,11pt]{article}

% Soporte para los acentos.
\usepackage[utf8]{inputenc}
\usepackage[T1]{fontenc}
% Idioma español.
\usepackage[spanish,mexico, es-tabla]{babel}
% Soporte de símbolos adicionales (matemáticas)
\usepackage{amsmath}
\usepackage[dvipsnames]{xcolor}
% Modificamos los márgenes del documento.                                       
\usepackage[lmargin=1.5cm,rmargin=1.5cm,top=1.5cm,bottom=1.5cm]{geometry}

\title{Facultad de Ciencias, UNAM \\ 
       Lenguajes de Programación\\ 
       Examen Parcial 3}
\author{Rubí Rojas Tania Michelle}
\date{04 de febrero de 2021}

\begin{document}
\maketitle

\begin{enumerate}
    % Ejercicio 1.
    \item Evalúa las siguientes expresiones usando cada uno de los pasos de 
    parámetros que se solicitan, debes de poner la última expresión a evaluar
    antes de dar el resultado final de la misma.
    \begin{verbatim}
    {with* {{i -1} {j -1}
              {swap {fun {x y}
                         {seqn {set tmp x}
                               {set x y}
                               {set y tmp}}}}
            {seqn {swap i j}
                  {- j i}}}}
    \end{verbatim}

    \begin{itemize}
        % Ejercicio 1.a
        \item Usando paso de parámetros por valor.
        \item 
        \textsc{Solución:} La expresión que debemos evaluar es 
        \texttt{\{seqn \{swap i j\} \{- j i\}\}}. La primera expresión con 
        la que nos encontramos es \texttt{\{swap i j\}}, la cual es una 
        función que recibe los parámetros $x,y$; y en la aplicación le 
        estamos pasando $i,j$. Por lo que los parámetros formales serán 
        $x,y$ y los reales serán $i,j$. Como vamos a usar paso por valor, 
        entonces \texttt{\{set tmp x\}} (con \texttt{tmp=x=-1}) es una 
        copia del valor $i=-1$. De igual forma, cuando hacemos 
        \texttt{\{set x y\}} (con \texttt{x=y=-1}) pasamos una copia del 
        valor de $y$
        
        Por último, el ambiente y la memoria son:
        \begin{table}[h]
            \parbox{.30\linewidth}{
            \centering
            \begin{tabular}{|c|c|}
            \hline
            \texttt{swap} & \texttt{2} \\
            \hline
            \texttt{j} & \texttt{1} \\
            \hline
            \texttt{i} & \texttt{0} \\
            \hline
            \end{tabular}
            }
            \hfill
            \parbox{.67\linewidth}{
            \centering
            \begin{tabular}{|c|c|}
            \hline
            \texttt{2} & \texttt{\{fun \{x y\} \{seqn \{set tmp x\}
                  \{set x y\}
                  \{set y tmp\}\}\}} \\
            \hline
            \texttt{1} & \texttt{-1} \\
            \hline
            \texttt{0} & \texttt{-1} \\
            \hline
            \end{tabular}
            }
        \end{table}

        % Ejercicio 1.b
        \item Usando paso de parámetros por referencia.
    \end{itemize}

    % Ejercicio 2.
    \item ¿Cuáles son las diferencias principales entre el paso de parámetros
    por necesidad y por nombre?

    % Ejercicio 3.
    \item Del siguiente código en \textsc{Racket}:
    \begin{verbatim}
    (define (filter-neg l)
      (cond
        [(empty? l) empty]
        [else 
          (if (< (first l) 0)
              (cons (first l) (filter-neg (rest l)))
              (filter-neg (rest l)))]))
    \end{verbatim}

    \begin{enumerate}
        % Ejercicio 3.a
        \item Convierte el código anterior a CPS.
        % Ejercicio 3.b
        \item ¿Qué regresa la función que convertiste a CPS cuando recibe la 
        lista \texttt{'(0 1 -1 0 -4 1 -2)}?
    \end{enumerate}

    % Ejercicio 4.
    \item Da la expresión asociada a la continuación y el resultado de dicha 
    expresión, para cada uno de los siguientes códigos:
    % Ejercicio 4.a
    \begin{verbatim}
    > (define c #f)
    > (+ 1 (+ 2 (+ 3 (+ (let/cc here 
                        (set! c here)
                        4) 5))))
    \end{verbatim}

    \textsc{Solución:}

    % Ejercicio 4.b
    \begin{verbatim}
    > (define c empty)
    > (+ 1 (+ 2 (+ 3 (+ (let/cc here (set! c here) 4) 5))))
    > (c 20)
    \end{verbatim}

    % Ejercicio 5.
    \item Da el juicio de tipo para la siguiente expresión, la cual está 
    implementada en el lenguaje \textsc{Racket}.
    \begin{verbatim}
    {let {x {+ 1 1}}
      {{fun {y} {+ 0 x}} {+ 2 2}}}
    \end{verbatim}

    % Ejercicio 6.
    \item Realiza la inferencia de tipos de la siguiente expresión, mencionándo
    al término de la inferencia, los tipos de cada una de las variables de la 
    función.
    \begin{verbatim}
    (define foo 
      (lambda (lst item)
        (cond 
          [(nempty? nlist) nempty]
          [(nequal? item (nfirst lst)) (nrest lst)]
          [else (ncons (nfirst lst) (foo (nrest lst) item))])))
    \end{verbatim}

    \textsc{Solución:} Primero, identificamos cada una de nuestras 
    sub-expresiones y las enumeramos.
    \begin{itemize}
        \item \fbox{1} \texttt{(lambda (lst item) (cond [(nempty? nlist) nempty]
        [(nequal? item (nfirst lst))} \\ 
        \texttt{(nrest lst)] [else (ncons (nfirst lst) (foo (nrest lst) item))]))}

        \item \fbox{2} \texttt{(cond [(nempty? nlist) nempty] [(nequal? item 
        (nfirst lst)) (nrest lst)]} \\ 
        \texttt{[else (ncons (nfirst lst) (foo (nrest lst) item))])}

        \item \fbox{3} \texttt{(nempty? nlist)}

        \item \fbox{4} \texttt{nempty}
        
        \item \fbox{5} \texttt{(nequal? item (nfirst lst))}

        \item \fbox{6} \texttt{(nfirst lst)}
        
        \item \fbox{7} \texttt{(nrest lst)}
        
        \item \fbox{8} \texttt{else}
        
        \item \fbox{9} \texttt{(ncons (nfirst lst) (foo (nrest lst) item))}

        \item \fbox{10} \texttt{(nfirst lst)}

        \item \fbox{11} \texttt{(foo (nrest lst) item)}

        \item \fbox{12} \texttt{(nrest lst)}
    \end{itemize}

    Luego, vamos a analizar el tipo de expresiones que encontramos.
    \begin{itemize}
        \item Para la cajita uno,
        \begin{align*}
            [[\;\fbox{1}\;]]
            &= [[\texttt{(lambda (lst item) (cond [(nempty? nlist) nempty]} \\ 
            &\; \; \; \; \; \; \; \texttt{[(nequal? item (nfirst lst)) 
            (nrest lst)]} \\
            &\; \; \; \; \; \; \; \texttt{[else (ncons (nfirst lst) 
            (foo (nrest lst) item))]))}]] \\ 
            &= [[\texttt{lst}]] \times [[\texttt{item}]] \rightarrow 
            [[\texttt{(cond [(nempty? nlist) nempty]} \\ 
            &\; \; \; \; \; \; \; \texttt{[(nequal? item (nfirst lst)) 
            (nrest lst)]} \\
            &\; \; \; \; \; \; \; \texttt{[else (ncons (nfirst lst) 
            (foo (nrest lst) item))])}]] \\ 
            &= [[\texttt{lst}]] \times [[\texttt{item}]] \rightarrow [[\;\fbox{2}\;]]
        \end{align*}
   
        \item Para la cajita dos, 
        \begin{align*}
            [[\;\fbox{2}\;]] 
            &= [[\texttt{(cond [(nempty? nlist) nempty] [(nequal? item 
            (nfirst lst)) (nrest lst)]} \\ 
            &\; \; \; \; \; \; \; \texttt{[else (ncons (nfirst lst) 
            (foo (nrest lst) item))])}]] \\ 
            &= [[\texttt{(cond [\;\fbox{3} \fbox{4}\;] [\;\fbox{5} \fbox{7}\;] 
            [\;\fbox{8} \fbox{9}\;])}]] \\
            &= [[\;\fbox{3} \rightarrow \fbox{4}\;]] \texttt{or} 
            [[\;\fbox{5} \rightarrow \fbox{7}\;]] \texttt{or}
            [[\;\fbox{8} \rightarrow \fbox{9}\;]]
        \end{align*}

        de donde 
        \begin{itemize}
            \item $[[\;\fbox{3}\;]] =$ \texttt{boolean}
            \item $[[\;\fbox{5}\;]] =$ \texttt{boolean}
            \item $[[\;\fbox{4}\;]] = [[\;\fbox{7}\;]] = [[\;\fbox{9}\;]]$
        \end{itemize}

        \item Para la cajita tres, 
        \begin{equation*}
            [[\;\fbox{3}\;]] = [[\texttt{(nempty? nlist)}]]
        \end{equation*}

        donde 
        \begin{itemize}
            \item $[[\texttt{(nempty? list)}]] = \texttt{boolean}$
            \item $[[\texttt{list}]] = \texttt{nlist}$
        \end{itemize}

        \item Para la cajita cuatro, 
        \begin{equation*}
            [[\;\fbox{4}\;]] = [[\texttt{nempty}]] = \texttt{nlist}
        \end{equation*}

        \item Para la cajita cinco,
        \begin{align*}
            [[\;\fbox{5}\;]]
            &= [[\texttt{(nequal? item (nfirst lst))}]] \\ 
            &= [[\texttt{(nequal? item \fbox{6})}]] 
        \end{align*}

        de donde 
        \begin{itemize}
            \item $[[\texttt{(nequal? item \fbox{6})}]] = \texttt{boolean}$
            \item $[[\texttt{item}]] = \texttt{number}$
            \item $[[\;\fbox{6}\;]] = [[\texttt{(nfirst lst)}]]$
            \begin{itemize}
                \item $[[\texttt{(nfirst lst)}]] =$ \texttt{number}
                \item $[[\texttt{lst}]] = \texttt{nlist}$
            \end{itemize}
        \end{itemize}

        \item Para la cajita siete, 
        \begin{equation*}
            [[\;\fbox{7}\;]] = [[\texttt{(nrest lst)}]]
        \end{equation*}

        donde 
        \begin{itemize}
            \item $[[\texttt{(nrest lst)}]] =$ \texttt{nlist}
            \item $[[\texttt{lst}]] = \texttt{nlist}$
        \end{itemize}

        \item Para la cajita ocho,
        \begin{equation*}
            [[8]] = [[\texttt{else}]] = [[\texttt{true}]] = \texttt{boolean}
        \end{equation*}

        \item Para la cajita nueve,
        \begin{align*}
            [[\;\fbox{9}\;]]
            &= [[\texttt{(ncons (nfirst lst) (foo (nrest lst) item))}]] \\ 
            &= [[\texttt{(ncons \fbox{10} \fbox{11})}]] 
        \end{align*}

        de donde 
        \begin{itemize}
            \item $[[\;\fbox{9}\;]] = \texttt{nlist}$
            \item $[[\;\fbox{10}\;]] = [[\;\fbox{6}\;]] = \texttt{nlist}$
            \item $[[\;\fbox{11} \;]] = [[\texttt{(foo (nrest lst) item)}]] = $
            $[[\texttt{(foo \fbox{12} item)}]] =$ \\ 
            $[[\;\fbox{12}\;]] \times [[\texttt{item}]] \rightarrow 
            [[\texttt{(foo (nrest lst) item)}]]$

            con 
            \begin{itemize}
                \item $[[\;\fbox{12}\;]] = [[\;\fbox{7}\;]] = \texttt{nlist}$
            \end{itemize}
        \end{itemize}
    \end{itemize}

    Por lo tanto, los tipos de las variables de la función son \texttt{lst = nlist}
    e \texttt{item = number}; por lo que el tipo de la función \texttt{foo} es 
    \begin{equation*}
        \texttt{foo: (nlist x number) -> nlist}
    \end{equation*}

    % Ejercicio 7.
    \item Utiliza el algoritmo de unificación visto en clase en la expresión
    \begin{verbatim}
    ((lambda (y) (* y (+ 0 0))) 1)
    \end{verbatim}

    \textsc{Solución:} Primero, identificamos cada una de nuestras 
    sub-expresiones y las enumeramos.
    \begin{itemize}
        \item \fbox{1} \texttt{((lambda (y) (* y (+ 0 0))) 1)}
        \item \fbox{2} \texttt{(lambda (y) (* y (+ 0 0)))}
        \item \fbox{3} \texttt{(* y (+ 0 0))}
        \item \fbox{4} \texttt{(+ 0 0)}
        \item \fbox{5} \texttt{0}
        \item \fbox{6} \texttt{0}
        \item \fbox{7} \texttt{1}
    \end{itemize}

    Luego, generamos las restricciones de tipo.
    \begin{itemize}
        \item $[[\;\fbox{1}\;]]$ = $[[\;\fbox{2}\;]] \rightarrow$
    \end{itemize}

    % Ejercicio 8.
    \item Da las sentencias de variables de tipo para las siguientes funciones 
    de \texttt{Racket}:
    \begin{itemize}
        % Ejercicio 8.a
        \item \texttt{list-length}

        \textsc{Solución:}
        \begin{equation*}
            \texttt{list-length: }\forall \alpha. \texttt{list}(\alpha) 
        \rightarrow \texttt{number}
        \end{equation*}

        % Ejercicio 8.b
        \item \texttt{fibonacci}
        
        \textsc{Solución:} 
        \begin{equation*}
            \texttt{fibonacci: } \forall n. \texttt{number}(n) \rightarrow 
            \texttt{number}
        \end{equation*}
    \end{itemize}

    % Ejercicio 9.
    \item Selecciona dos características de la siguiente lista del Paradigma 
    Orientado a Objetos:
    \begin{itemize}
        \item Herencia
        \item Encapsulamiento de información
        \item Abstracción
        \item Modularidad
    \end{itemize}

    Ahora define y explícalas, usando ejemplos.

    \textsc{Solución:}
    \begin{itemize}
        % Ejercicio 9.a
        \item Herencia

        La herencia es el procedimiento por el cual una clase hereda los 
        atributos y métodos de otra clase. La clase cuyas propiedades y métodos 
        se heredan se conoce como clase Padre. Y la clase que hereda las 
        propiedades de la clase padre es la clase hija.

        Ejemplo: Supongamos que tenemos alumnos universitarios. Algunos son 
        alumnos normales, otros son de intercambio y otros son becarios. 
        Probablemente tendremos una clase \texttt{Alumno} con una serie de 
        métodos como \texttt{asistir\_a\_clase()}, \texttt{hacer\_examen()}, 
        etc., y una serie de atributos como \texttt{nombre}, \texttt{número de 
        cuenta}, etc.; que son comúnes a todos los alumnos, pero hay operaciones 
        que son diferentes en cada tipo de alumno como 
        \texttt{pagar\_mensualidad()} (los becarios no pagan) o 
        \texttt{matricularse()} (los de intercambio se matriculan en su 
        universidad de origen). En este caso, la clase Padre sería 
        \texttt{Alumno} y las clases hijas serán \texttt{AlumnoNormal},
        \texttt{AlumnoIntercambio} y \texttt{AlumnoBecado}.

        % Ejercicio 9.b
        \item Abstracción

        Una clase abstracta es aquella en la que no podemos instanciar objetos.
        Expresa las características específicas de un objeto, aquellas que lo 
        distinguen de los demás tipos y que logran definir límites conceptuales
        respecto a quién está haciéndo dicha abstracción.

        Ejemplo: Supongamos que queremos aplicar la Abstracción a las Aves. 
        Un \texttt{Ave} es sólo un concepto abstracto que no puede instanciarse.
        Existen muchas aves que heredan sus características (como el pico, las 
        plumas, las alas, las patas, volar, etc) y ellas sí pueden existir 
        por sí mismos. Un pájaro o un pato serían ejemplos de aves que sí 
        pueden existir de la clase abstracta \texttt{Ave}.
    \end{itemize}

    % Ejercicio 10.
    \item Se tiene el siguiente código que intenta dar una versión 
    \textit{memoizada} de la función que calcula el $n-$ésimo número de la 
    sucesión de \texttt{Tribonacci}:
    \begin{verbatim}
    (define (tribonacci n)
      (if (< n 3)
          1
          (+ (tribonacci (- n 1)) (tribonacci (- n 2)) (tribonacci (- n 3)))))
    \end{verbatim}
    \begin{verbatim}
    (define (tribonacci-memo n tabla)
      (let ([res (hash-ref tabla n 'ninguno)])
        (cond
          [(equal? res 'ninguno)
           (hash-set! tabla n (tribonacci n))
           (hash-ref tabla n)]
          [else res])))
    \end{verbatim}

    \begin{enumerate}
        % Ejercicio 10.a
        \item ¿Por qué la función \texttt{tribonacci-memo} del código anterior 
        no hace uso correcto de la técnica de memoización?

        \textsc{Solución:} Esta función no hace uso correcto de la memoización
        porque gracias a la línea 
        \begin{center}
            \texttt{(hash-set! tabla n (tribonacci n))}
        \end{center}

        efectivamente vamos agregando nuevos registros a la tabla, pero como 
        el valor lo obtenemos llamando a la función \texttt{(tribonacci n)}, 
        entonces se están generando sus respectivas llamadas a función 
        repetidas, es decir, al utilizar la función \texttt{(tribonacci n)} 
        estamos volviendo a calcular valores que ya habíamos obtenido antes 
        (lo cual queremos evitar usando memoización).

        % Ejercicio 10.b
        \item Describir cómo se corregiría este código para que emplee bien la 
        técnica de memoización.
        
        \textsc{Solución:} Primero, debemos cambiar la línea 
        \begin{center}
            \texttt{(hash-set! tabla n (tribonacci n))}    
        \end{center}

        por la definición de una variable \texttt{nuevo}, la cual debe calcular 
        el valor para el parámetro realizando la operación
        \begin{center}
            \texttt{(+ (tribonacci (- n 1)) (tribonacci (- n 2)) 
            (tribonacci (- n 3)))}
        \end{center}
        
        y después debemos modificar la línea
        \begin{center}
            \texttt{(hash-ref tabla n)}
        \end{center}

        para poder asignar este valor a la tabla y posteriormente regresarlo.
        
        Por lo tanto, la función modificada queda como:
        \begin{verbatim}
        (define (tribonacci-memo n tabla)
          (let ([res (hash-ref tabla n 'ninguno)])
            (cond
              [(equal? res 'ninguno)
               (define nuevo
                 (+ (tribonacci (- n 1)) 
                    (tribonacci (- n 2)) 
                    (tribonacci (- n 3))))
               (hash-set! tabla n nuevo)
               nuevo]
              [else res])))
        \end{verbatim}
    \end{enumerate}

\end{enumerate}

\end{document}
