\documentclass[letterpaper,11pt]{article}

% Soporte para los acentos.
\usepackage[utf8]{inputenc}
\usepackage[T1]{fontenc}    
% Idioma español.
\usepackage[spanish,mexico, es-tabla]{babel}
% Soporte de símbolos adicionales (matemáticas)
\usepackage{multirow}
\usepackage{amsmath}
\usepackage{amssymb}
\usepackage{amsthm}
\usepackage{amsfonts}
\usepackage{mathtools}
\usepackage{latexsym}
\usepackage{enumerate}
\usepackage{ragged2e}
\usepackage{graphicx}
\usepackage{hyperref}
\usepackage{xcolor}
% Modificamos los márgenes del documento.
\usepackage[lmargin=1cm,rmargin=1cm,top=1.5cm,bottom=1.5cm]{geometry}

\title{Facultad de Ciencias, UNAM \\ 
       Lenguajes de Programación \\ 
       Tarea 3}
\author{Hernández Salinas Óscar \\ 
        Rubí Rojas Tania Michelle }
\date{26 de octubre de 2020}

\begin{document}
\maketitle

\begin{enumerate}
    % Ejercicio 1.
    \item Evalúa las siguientes expresiones usando \texttt{a) alcance estático}
    y \texttt{b) alcance dinámico}. Es necesario que muestres el ambiente de 
    evaluación final en forma de pila y en forma de lista en cada caso.
    \begin{enumerate}
        % Ejercicio 1.a
        \item 
        \begin{verbatim}
        {with {a 2} 
           {with {b 3} 
              {with {foo {fun {x} {- {+ a b} x}}} 
                 {with {a -2} 
                    {with {b -3} 
                       {with {foo {fun {x} {+ {- a b} x}}} 
                          {foo -10}}}}}}}
        \end{verbatim}

        \textsc{Solución:}

        % Ejercicio 1.b
        \item 
        \begin{verbatim}
        {with {foo {fun {x} {+ x {foo {- x 1}}}}} 
           {foo 10}}
        \end{verbatim}

        \textsc{Solución:}

        % Ejercicio 1.c
        \item 
        \begin{verbatim}
        {with {x 2} 
           {with {foo {fun {a} {+ x 2}}} 
              {with {y 3} 
                 {with {foo {fun {b} {- y b}}} 
                    {with {x 4} 
                       {with {goo {fun {b} {+ {foo x} {foo y}}}} {goo 3}}}}}}}
        \end{verbatim}

        \textsc{Solución:}
    \end{enumerate}

    % Ejercicio 2.
    \item Las primeras versiones del lenguaje \textsc{Lisp} hacían uso de 
    alcance dinámico y sus diseñadores se negaban a cambiárlo debido a una 
    gran ventaja que traía consigo el uso de este tipo de alcance. Con base 
    en los resultados obtenidos en el inciso \texttt{(b)} del ejercicio 
    anterior, menciona esta ventaja.

    \textsc{Solución:}

    % Ejercicio 3.
    \item En clase se revisó que la función de \textit{sustitución} es 
    ineficiente, ya que en el peor caso es de orden cuadrático en relación 
    al tamaño del programa (considerando el tamaño del programa como el 
    número de nodos en el árbol de sintaxis abstracta). Por otro lado, se 
    expuso una alternativa a este algoritmo de sustitución usando 
    ambientes. Sin embargo, el implementar un ambiente usando una pila no 
    parece ser mucho más eficiente.
    \begin{enumerate}
        % Ejercicio 3.a
        \item Da un programa que ilustre la no linealidad de la implementación
        basada en pilas y explica brevemente por qué su ejecución en tiempo no
        es lineal con respecto al tamaño de la entrada.

        \textsc{Solución:}

        % Ejercicio 3.b
        \item Describe una estructura de datos para un ambiente que un 
        intérprete de FWAE pueda usar para mejorar su complejidad. Muestra 
        cómo el intérprete usaría esta nueva estructura de datos. Indica 
        además, cuál es la nueva complejidad del intérprete (análisis del 
        peor caso).

        \textsc{Solución:}
    \end{enumerate}
\end{enumerate}

\end{document}
