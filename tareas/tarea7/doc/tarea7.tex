\documentclass[letterpaper,11pt]{article}

% Soporte para los acentos.
\usepackage[utf8]{inputenc}
\usepackage[T1]{fontenc}
% Idioma español.
\usepackage[spanish,mexico, es-tabla]{babel}
% Soporte de símbolos adicionales (matemáticas)
\usepackage{amsmath}
\usepackage[dvipsnames]{xcolor}
% Modificamos los márgenes del documento.                                       
\usepackage[lmargin=1.5cm,rmargin=1.5cm,top=1.5cm,bottom=1.5cm]{geometry}

\usepackage{graphicx}
\usepackage{alltt}

\usepackage{amssymb}

\usepackage{bussproofs}
\EnableBpAbbreviations

\title{Facultad de Ciencias, UNAM \\ 
       Lenguajes de Programación\\ 
       Tarea 7}
\author{ Rodríguez Campos Erick Eduardo\\ Rubí Rojas Tania Michelle}
\date{31 de enero de 2021}


\begin{document}
\maketitle
                                                                                %
\begin{enumerate}
    \item Da la derivación de las siguientes expresiones usando las reglas de 
    semántica operacional para $FAE$, vistas en clase:
    \begin{itemize}
        \item [$(a)$] \texttt{\{- \{\{fun \{x\} x\} 2 \} \{+ 3 5\}\}}
        \item [$(b)$] \texttt{\{\{\{fun \{x\} \{fun \{y\}\{+ x y\}\}\}2\}3\}}
    \end{itemize}
    \item Realiza el juicio de tipo para cada una de las siguientes expresiones, 
    usa las reglas visitas en clase.
    \begin{itemize}
        \item [$(a)$] \begin{verbatim}{with {a : number 2}
    {+ a 2}} 
        \end{verbatim}
        \begin{prooftree}
            \AxiomC{$\varnothing \vdash 2:number$}
            \AxiomC{$[a\leftarrow number]\vdash a:number$}
            \AxiomC{$[a\leftarrow number]\vdash 2:number$}
            \BinaryInfC{$[a\leftarrow number]\vdash \{+\; a\; 2\}:number$}
            \BinaryInfC{$\varnothing \vdash \{with \{a:number\;2\}\;\{+\; a\; 2\}\}:number$}
        \end{prooftree}
        \item [$(b)$] \begin{verbatim}{fun {x : number} : number {+ x 2}}
        \end{verbatim}
        \begin{prooftree}
            \AxiomC{$[x\leftarrow number]\vdash x:number$}
            \AxiomC{$[x\leftarrow number]\vdash 2:number$}
            \BinaryInfC{$[x\leftarrow number]\vdash \{+\;x\;2\}:number$}
            \UnaryInfC{$\varnothing \vdash \{fun \{x:number\}:number\{+\;x\;2\}\}:\{number \rightarrow number\}$}
        \end{prooftree}
        \item [$(c)$] \begin{verbatim}{{fun {x} {+ x 2}} {+ 3 4}}
        \end{verbatim}
        \begin{prooftree}
            \AxiomC{$[x\leftarrow number]\vdash x:number$}
            \AxiomC{$[x\leftarrow number]\vdash 2:number$}
            \BinaryInfC{$[x\leftarrow number]\vdash \{+\;x\;2\}:number$}
            \UnaryInfC{$\varnothing \vdash \{fun \{x:number\}:number\{+\;x\;2\}\}:\{number \rightarrow number\}$}
            \AxiomC{$\varnothing \vdash 3:number$}
            \AxiomC{$\varnothing \vdash 2:number$}
            \BinaryInfC{$\varnothing \vdash \{+\; 3\; 4\}:number$}
            \BinaryInfC{$\varnothing \vdash \{\{fun \{x:number\}:number\{+\; x\; 2\}\}\; \{+\; 3\; 4\}\}:number$} 
        \end{prooftree}
        \item [$(d)$]\begin{verbatim}{with {f : {number -> number} {fun {x : number} : number {+ x 2}}} 
    {f {+ 3 4}}}
        \end{verbatim}
        \begin{prooftree}
            \AxiomC{A}
            \AxiomC{B}
            \BinaryInfC{C}
            \UnaryInfC{D}
            \AxiomC{E}
            \AxiomC{F}
            \AxiomC{G}
            \BinaryInfC{H}
            \BinaryInfC{I}
            \BinaryInfC{J}
        \end{prooftree}
        Donde:
        \begin{itemize}
            \item [$A$] es $[x\leftarrow number]\vdash x:number$
            \item [$B$] es $[x\leftarrow number]\vdash 2:number$
            \item [$C$] es $[x\leftarrow number]\vdash \{+\; x\; 2\}:number$
            \item [$D$] es $\varnothing \vdash \{fun\{x:number\}:number\{+\;x\;2\}\}:\{number\rightarrow number\}$
            \item [$E$] es $[f\leftarrow\{number \rightarrow number\}] \vdash f:\{number\rightarrow number\}$
            \item [$F$] es $[f \leftarrow \{number \rightarrow number\}]\vdash 3:number$
            \item [$G$] es $[f \leftarrow \{number \rightarrow number\}]\vdash 2:number$
            \item [$H$] es $[f\leftarrow \{number\rightarrow number\}]\vdash\{+\;3\;4\}:number$
            \item [$I$] es $[f\leftarrow\{number\rightarrow number\}]\vdash \{f\{+\;3\;4\}\}:number$
            \item [$J$] es $\varnothing \vdash \{with \{f:\{number\rightarrow number\}\{fun \;\{x:number\}:number\;\{+\;x\;2\}\}\}\;\{f\{+\;3\;4\}\}\}:number$
        \end{itemize}
        \item [$(e)$]\begin{verbatim}{with {g {fun {x} {x 4}}}
    {g {fun {y} {- y 2}}}}
        \end{verbatim}
        \item [$(f)$]\begin{verbatim}{rec {f : {number -> number}
    {fun {x : number} : number
        {if0 x 1 {* n {f {- n 1}}}}}}
    {f 5}}
        \end{verbatim}
    \end{itemize}
    
    % Ejercicio 3.
    \item Para cada una de las siguientes expresiones, realiza su inferencia 
    de tipos generando las restricciones de tipo correspondientes
    \begin{enumerate}
        % Ejercicio 3.a
        \item 
        \begin{verbatim} (define (potencia a b)
    (if (zero? b)
        1
        (* a (potencia a (sub1 b)))))
        \end{verbatim}
        
        \textsc{Solución:} Primero, identificamos cada una de nuestras 
        sub-expresiones y las enumeramos.
        \begin{itemize}
            \item \fbox{1} \texttt{(if (zero? b) 1 (* a (potencia a (sub1 b))))}

            \item \fbox{2} \texttt{(zero? b)}

            \item \fbox{3} \texttt{1}

            \item \fbox{4} \texttt{(* a (potencia a (sub1 b)))}

            \item \fbox{5} \texttt{(potencia a (sub1 b))}

            \item \fbox{6} \texttt{(sub1 b)}
        \end{itemize}
        
        Luego, vamos a analizar el tipo de expresiones que encontramos.
        \begin{itemize}
            \item Para la primer cajita, 
            \begin{align*}
                [[ \; \fbox{1} \; ]]
                &= [[ \; \texttt{(if (zero? b) 1 (* a (potencia a (sub1 b))))} \; ]] \\
                &= [[ \; \texttt{(if \fbox{2} \fbox{3} \fbox{4})} \; ]]
            \end{align*}
            
            de donde 
            \begin{itemize}
                \item $[[$ \fbox{1} $]]$ = $[[$ \fbox{3} $]]$
                \item $[[$ \fbox{1} $]]$ = $[[$ \fbox{4} $]]$
                \item $[[$ \fbox{3} $]]$ = $[[$ \fbox{4}$]]$ 
                \item $[[$ \fbox{2} $]]$ = \texttt{boolean}
            \end{itemize}
            
            \item Para la segunda cajita, 
            \begin{align*}
                [[ \; \fbox{2} \; ]]
                &= [[\texttt{(zero? b)}]]
            \end{align*}
            
            de donde 
            \begin{itemize}
                \item $[[$\texttt{(zero? b)}$]] =$ \texttt{boolean}
                \item $[[$\texttt{b}$]] =$ \texttt{number}
            \end{itemize}
            
            \item Para la tercer cajita, 
            \begin{align*}
                [[ \; \fbox{3} \; ]] = [[1]] = \texttt{number}
            \end{align*}
            
            \item Para la cuarta cajita, 
            \begin{align*}
                [[\; \fbox{4} \;]]
                &= [[\texttt{(* a (potencia a (sub1 b)))}]] \\
                &= [[\texttt{(* a \fbox{5}}]]
            \end{align*}
            
            de donde 
            \begin{itemize}
                \item $[[\; \fbox{4}\;]]$ = \texttt{number}
                \item $[[$\texttt{a}$]] = $ \texttt{number}
                \item $[[\; \fbox{5} \;]] =$ \texttt{number}
            \end{itemize}
            
            \item Para la quinta cajita,
            \begin{align*}
                [[ \; \fbox{5} \; ]]
                &= [[\texttt{(potencia a (sub1 b))}]] \\
                &= [[\texttt{(potencia a \fbox{6})}]]
            \end{align*}

            de donde $[[\texttt{a} \rightarrow \fbox{6}]]$
            
            \item Para la sexta cajita,
            \begin{equation*}
                [[\; \fbox{6} \;]] = [[\texttt{(sub1 b)}]]
            \end{equation*}

            de donde 
            \begin{itemize}
                \item $[[$\texttt{(sub1 b)}$]] =$ \texttt{number}
                \item $[[$\texttt{b}$]] =$ \texttt{number}
            \end{itemize}
        \end{itemize}
        
        Por lo tanto, el tipo de la función \texttt{potencia} es 
        \begin{center}
            \texttt{potencia: number number -> number}
        \end{center}
        
        donde $a$ y $b$ son ambos \texttt{number}.
        
        % Ejercicio 3.b
        \item \begin{verbatim}(define (suma l)
    (if (nempty? l)
        0
        (ncons (nfirst l) (suma (nrest l)))))
        \end{verbatim}

        \textsc{Solución:} Primero, identificamos cada una de nuestras 
        sub-expresiones y las enumeramos.
        \begin{itemize}
            \item \fbox{1} \texttt{(if (nempty? l) 0 
            (ncons (nfirst l) (suma (nrest l))))}

            \item \fbox{2} \texttt{(nempty? l)}
            
            \item \fbox{3} \texttt{0}
            
            \item \fbox{4} \texttt{(ncons (nfirst l) (suma (nrest l)))}

            \item \fbox{5} \texttt{(nfirst l)}

            \item \fbox{6} \texttt{(suma (nrest l))}

            \item \fbox{7} \texttt{(nrest l)}
        \end{itemize}

        Luego, vamos a analizar el tipo de expresiones que encontramos.
        \begin{itemize}
            \item Para la primer cajita, 
            \begin{align*}
                [[\; \fbox{1} \;]] 
                &= [[ \texttt{(if (nempty? l) 0 
                (ncons (nfirst l) (suma (nrest l))))}]] \\ 
                &= [[ \texttt{(if \fbox{2} \fbox{3} \fbox{4})} ]]
            \end{align*}

            de donde 
            \begin{itemize}
                \item $[[\; \fbox{1} \; ]]$ $=$ $[[\; \fbox{3} \;]]$
                \item $[[\; \fbox{1} \; ]]$ $=$ $[[\; \fbox{4} \;]]$
                \item $[[\; \fbox{3} \; ]]$ $=$ $[[\; \fbox{4} \;]]$
                \item $[[\; \fbox{2} \; ]]$ $=$ \texttt{boolean}
            \end{itemize}

            \item Para la segunda cajita, 
            \begin{equation*}
                [[\; \fbox{2} \;]] = [[\; \texttt{(nempty? l)} \;]] 
            \end{equation*}

            de donde 
            \begin{itemize}
                \item $[[$\texttt{(nempty? l)}$]]$ = \texttt{boolean}
                \item $[[$\texttt{b}$]]$ = \texttt{nlist}
            \end{itemize}

            \item Para la tercer cajita,
            \begin{equation*}
                [[\; \fbox{3} \;]] 
                = [[\texttt{0} ]]
                = \texttt{number}
            \end{equation*}

            \item Para la cuarta cajita, 
            \begin{align*}
                [[\; \fbox{4} \;]]
                &= [[\; \texttt{(ncons (nfirst l) (suma (nrest l)))} \;]] \\ 
                &= [[\texttt{(ncons \fbox{5} \fbox{6})}]]
            \end{align*}

            de donde 
            \begin{itemize}
                \item $[[\; \fbox{4} \;]] =$ \texttt{nlist}
                \item $[[\; \fbox{5} \;]] = [[\texttt{(nfirst l)}]] =$
                \texttt{number}

                \item $[[\; \fbox{6} \;]] = [[\texttt{(suma (nrest l))}]]$
                \begin{itemize}
                    \item $[[\texttt{suma}]] = [[\texttt{(nrest l)}]] 
                    \rightarrow [[\;\fbox{6}\;]] = [[\;\fbox{7}\;]] 
                    \rightarrow [[\;\fbox{6}\;]] = $ \texttt{nlist}
                    \begin{itemize}
                        \item $[[\texttt{(nrest l)}]] = \texttt{nlist}$
                        \item $[[\texttt{l}]] = \texttt{nlist}$
                    \end{itemize}
                \end{itemize}
            \end{itemize}
        \end{itemize}

        Sin embargo, por el análisis de la primer cajita tenemos que 
        $[[\; \fbox{3} \;]] = [[\; \fbox{4} \;]]$, pero 
        \begin{equation*}
            [[\; \fbox{3} \;]] = \texttt{number} \neq 
            \texttt{nlist} = [[\; \fbox{4} \;]]
        \end{equation*}

        Por lo tanto, obtenemos una contradicción.

        % Ejercicio 3.c
        \item \begin{verbatim}(define (nfilter p l)
    (cond
        [(nempty? l) nempty]
        [(p (nfirst l)) (ncons (nfirst l) (nfilter p (nrest l)))]
        [else (nfilter p (nrest l))]))
        \end{verbatim}

        \textsc{Solución:} Primero, identificamos cada una de nuestras 
        sub-expresiones y las enumeramos.
        \begin{itemize}
            \item \fbox{1} \texttt{(cond [(nempty? l) nempty] 
            [(p (nfirst l)) (ncons (nfirst l) (nfilter p (nrest l)))] 
            [else (nfilter p (nrest l))])}

            \item \fbox{2} \texttt{(nempty? l)}

            \item \fbox{3} \texttt{nempty}

            \item \fbox{4} \texttt{(p (nfirst l))}

            \item \fbox{5} \texttt{(nfirst l)}

            \item \fbox{6} \texttt{(ncons (nfirst l) (nfilter p (nrest l)))}

            \item \fbox{7} \texttt{(nfirst l)}

            \item \fbox{8} \texttt{(nfilter p (nrest l))}

            \item \fbox{9} \texttt{(nrest l)}

            \item \fbox{10} \texttt{else}

            \item \fbox{11} \texttt{(nfilter p (nrest l))}

            \item \fbox{12} \texttt{(nrest l)}
        \end{itemize}

        Luego, vamos a analizar el tipo de expresiones que encontramos.
        \begin{itemize}
            \item Para la primer cajita, 
            \begin{align*}
                [[\;\fbox{1}\;]] 
                &= [[\texttt{(cond [\;\fbox{2} \fbox{3}\;] 
                [\;\fbox{4} \fbox{6}\;] [\;\fbox{10} \fbox{11}\;])}]] \\ 
                &= [[\;\fbox{2}\;]] \rightarrow [[\;\fbox{3}\;]] \texttt{or}
                [[\;\fbox{4}\;]] \rightarrow [[\;\fbox{6}\;]] \texttt{or}
                [[\;\fbox{10}\;]] \rightarrow [[\;\fbox{11}\;]]
            \end{align*}

            de donde 
            \begin{itemize}
                \item $[[\;\fbox{2}\;]] =$ \texttt{boolean}
                \item $[[\;\fbox{4}\;]] =$ \texttt{boolean}
                \item $[[\;\fbox{3}\;]] = [[\;\fbox{6}\;]] = [[\;\fbox{11}\;]]$
            \end{itemize}

            \item Para la segunda cajita, 
            \begin{equation*}
                [[\;\fbox{2}\;]] = \texttt{(nempty? l)}
            \end{equation*}

            de donde 
            \begin{itemize}
                \item $[[\texttt{(nempty? l)}]] =$ \texttt{boolean}
                \item $[[\texttt{b}]] =$ \texttt{nlist}
            \end{itemize}

            \item Para la tercer cajita, 
            \begin{equation*}
                [[\;\fbox{3}\;]] = \texttt{nempty} = \texttt{nlist} 
            \end{equation*}
            \item Para la cuarta cajita,
            \begin{align*}
                [[\;\fbox{4}\;]] 
                &= \texttt{(p (nfirst l))} \\ 
                &= \texttt{(p \fbox{5})}
            \end{align*}

            donde $[[\texttt{p}]] = [[\;\fbox{5}\;]] \rightarrow 
            [[\;\fbox{4}\;]]$

            \item Para la quinta cajita,
            \begin{equation*}
                [[\;\fbox{5}\;]] 
                = \texttt{(nfirst l)}
            \end{equation*}

            de donde 
            \begin{itemize}
                \item $[[(\texttt{(nfirst l)})]]$ = \texttt{number}
                \item $[[\texttt{l}]]$ = \texttt{nlist}
            \end{itemize}

            \item Para la sexta cajita, 
            \begin{align*}
                [[\;\fbox{6}\;]] 
                &= [[\texttt{(ncons (nfirst l) (nfilter p (nrest l)))}]] \\
                &= [[\texttt{(ncons \fbox{7} \fbox{8})}]]
            \end{align*}

            de donde 
            \begin{itemize}
                \item $[[\;\fbox{6}\;]] = \texttt{nlist}$
                \item $[[\;\fbox{7}\;]] = [[\;\fbox{5}\;]] =\texttt{number}$
                \item $[[\;\fbox{8}\;]] = \texttt{(nfilter p (nrest l))} = 
                \texttt{(nfilter p \fbox{9})}$
                \begin{itemize}
                    \item $[[\texttt{nfilter}]] = [[\texttt{p}]] [[\;\fbox{9}\;]]
                    \rightarrow [[\;\fbox{8}\;]] = \texttt{nlist}$
                \end{itemize}
            \end{itemize}

            \item Para la novena cajta,
            \begin{equation*}
                [[\;\fbox{9}\;]] = \texttt{(nrest l)}
            \end{equation*}

            de donde 
            \begin{itemize}
                \item $[[\texttt{(nrest l)}]] = $ \texttt{nlist}
                \item $[[\texttt{l}]] = $ \texttt{nlist}
            \end{itemize}

            \item Para la décima cajita, 
            \begin{equation*}
                [[\;\fbox{10}\;]] = [[\texttt{else}]] = [[\texttt{true}]] = 
                \texttt{boolean}
            \end{equation*}

            \item Para la undécima cajita, 
            \begin{equation*}
                [[\;\fbox{11}\;]] = [[\;\fbox{8}\;]] = \texttt{nlist}
            \end{equation*}

            \item Para la duodécima cajita, 
            \begin{equation*}
                [[\;\fbox{12}\;]] = [[\;\fbox{9}\;]] = \texttt{nlist}
            \end{equation*}
        \end{itemize}

        Por lo tanto, el tipo de la función \texttt{nfilter} es 
        \begin{center}
            \texttt{nfilter: (number -> boolean) nlist -> nlist}
        \end{center}

        donde $p$ es una función del tipo \texttt{(number -> boolean)}
        y $l$ es del tipo \texttt{nlist}.
    \end{enumerate}


    \item Usando el algoritmo de unificación, muestra la inferencia de tipos de las siguientes expresiones:
    \begin{itemize}
        \item [$(a)$] $(\lambda\; (x)\; (x\; 2\; 3))$
        \item [$(b)$] $((\lambda\; (x)\; (*\; x\; 2))\; (+\; 2\; 3))$
    \end{itemize}
    \item Indica si el sistema de Macros de \texttt{RACKET} y \texttt{C} es \textit{higiénico} o no. Justifica con un pequeño programa que haga uso de macros.
\end{enumerate}

\end{document}
