\documentclass[letterpaper,11pt]{article}

% Soporte para los acentos.
\usepackage[utf8]{inputenc}
\usepackage[T1]{fontenc}    
% Idioma español.
\usepackage[spanish,mexico, es-tabla]{babel}
% Soporte de símbolos adicionales (matemáticas)
\usepackage{multirow}
\usepackage{amsmath}
\usepackage{amssymb}
\usepackage{amsthm}
\usepackage{amsfonts}
\usepackage{mathtools}
\usepackage{latexsym}
\usepackage{enumerate}
\usepackage{ragged2e}
\usepackage{graphicx}
\usepackage{hyperref}
\usepackage{xcolor}
% Modificamos los márgenes del documento.
\usepackage[lmargin=1cm,rmargin=1cm,top=1.5cm,bottom=1.5cm]{geometry}

\title{Facultad de Ciencias, UNAM \\ 
       Lenguajes de Programación \\ 
       Tarea 2}
\author{Hernández Salinas Óscar \\ 
        Rubí Rojas Tania Michelle }
\date{19 de octubre de 2020}

\begin{document}
\maketitle

\begin{enumerate}
    % Ejercicio 1.
    \item Define las siguientes funciones sobre expresiones del lenguaje WAE:
    \begin{enumerate}
        % Ejercicio 1.a
        \item La función \texttt{libres}: \texttt{WAE} $\rightarrow$ 
        \texttt{(listof symbol)} que dada una expresión de tipo
        \texttt{WAE} devuelve una lista con los identificadores libres (sin 
        repeticiones) contenidos en ésta.

        \textsc{Solución:}
        \begin{verbatim}
        (define (libres expr)
          (type-case WAE expr
            [id (i) (list i)]
            [num (n) '()]
            [add (lhs rhs) (union (libres lhs) (libres rhs))]
            [sub (lhs rhs) (union (libres lhs) (libres rhs))]
            [with (id value body)
                  (union (libres-aux value (list id)) 
                         (libres-aux body (list id)))]))

        (define (libres-aux expr lst)
          (type-case WAE expr
            [id (i) (if (not (member? i lst))
                        (list i)
                        '())]
            [num (n) '()]
            [add (lhs rhs) (union (libres-aux lhs lst) (libres-aux rhs lst))]
            [sub (lhs rhs) (union (libres-aux lhs lst) (libres-aux rhs lst))]
            [with (id value body)
                  (union (libres-aux value (union lst (list id)))
                         (libres-aux body (union lst (list id))))]))
        \end{verbatim}

        \newpage
        % Ejercicio 1.b
        \item La función \texttt{ligadas}: \texttt{WAE} $\rightarrow$
        \texttt{(listof symbol)} que dada una expresión de tipo
        \texttt{WAE} devuelve una lista con identificadores ligados (sin
        repeticiones) contenidos en ésta.

        \textsc{Solución:}
        \begin{verbatim}
        (define (ligadas expr)
          (type-case WAE expr
            [id (i) '()]
            [num (n) '()]
            [add (lhs rhs) (union (ligadas lhs) (ligadas rhs))]
            [sub (lhs rhs) (union (ligadas lhs) (ligadas rhs))]
            [with (id value body)
                  (union (ligadas-aux value (list id)) 
                         (ligadas-aux body (list id)))]))
        
        (define (ligadas-aux expr lst)
          (type-case WAE expr
            [id (i) (if (member? i lst)
                        (list i)
                        '())]
            [num (n) '()]
            [add (lhs rhs) (union (ligadas-aux lhs lst) (ligadas-aux rhs lst))]
            [sub (lhs rhs) (union (ligadas-aux lhs lst) (ligadas-aux rhs lst))]
            [with (id value body)
                  (union (ligadas-aux value (union lst (list id))) 
                         (ligadas-aux body (union lst (list id))))]))
        \end{verbatim}

        % Ejercicio 1.c
        \item La función \texttt{de-ligado}: \texttt{WAE} $\rightarrow$ 
        \texttt{(listof symbol)} que dada una expresión de tipo \texttt{WAE} 
        devuelve una lista con identificadores de ligado (sin repeticiones) 
        contenidos en ésta.

        \textsc{Solución:}
        \begin{verbatim}
        (define (de-ligado expr) 
          (type-case WAE expr
            [id (i) '()]
            [num (n) '()]
            [add (lhs rhs) (union (de-ligado lhs) (de-ligado rhs))]
            [sub (lhs rhs) (union (de-ligado lhs) (de-ligado rhs))]
            [with (id value body)
                  (union (list id) (de-ligado value) (de-ligado body))]))
        \end{verbatim}
    \end{enumerate} 

    % Ejercicio 2.
    \item Sea $e$ una expresión del lenguaje \texttt{WAE}. Suponiéndo que 
    \texttt{(libres e) = '()}, demostrar o dar un contraejemplo de la siguiente 
    desigualdad.
    \begin{center}
        \texttt{(length (ligada e)) $\leq$ (length (de-ligado e))}
    \end{center}

    \begin{proof}
        Sea $e$ la siguiente expresión del lenguaje \texttt{WAE}
        \begin{center}
            \texttt{\textbf{\{with \{\textcolor{blue}{a} 17\} 
            \{+ \textcolor{red}{a} \{+ \textcolor{red}{a} \{+ \textcolor{red}{a} 
            \{+ \textcolor{red}{a} \textcolor{red}{a}\}\}\}\}\}}}
        \end{center}

        donde el símbolo de color \textcolor{blue}{azul} es una variable de 
        \textbf{de-ligado} y los símbolos de color \textcolor{red}{rojo} son 
        variables \textbf{ligadas}. Notemos, además, que no tenemos variables 
        \textbf{libres}. 

        \newpage
        Así, 
        \begin{center}
            \texttt{(length (ligada e)) $= 5 \not \leq 1$ (length (de-ligado e))}
        \end{center}

        Por lo tanto, la desigualdad \texttt{(length (ligada e)) $\leq$ 
        (length (de-ligado e))} es falsa.

    \end{proof}

    % Ejercicio 3.
    \item Realiza las siguientes sustituciones cuidando el alcance de las 
    variables correspondientes. Indica para cada expresión los identificadores 
    libres, de ligado y ligados. 
    
    De color \textcolor{blue}{azul} tenemos los identificadores \textbf{de-ligado} , de 
    color \textcolor{red}{rojo} los identificadores \textbf{ligados} y de \textcolor{green}{verde}
    los identificadores \textbf{libres}
    \begin{enumerate}
        % Ejercicio 3.a
        \item \texttt{\textbf{\{with \{w \{- u 8\}\} \{with \{v 5\} 
        \{+ w \{+ y x\}\}\}\} {[}x := \{+ u v\}{]}}}

        \textsc{Solución:}
        \texttt{\textbf{\{with \{\textcolor{blue}{w} \{- \textcolor{green}{u} 8\}\} \{with 
        \{\textcolor{blue}{v} 5\} \{+ \textcolor{red}{w} \{+ \textcolor{green}{y} 
        \{+ \textcolor{green}{u} \textcolor{red}{v}\}\}\}\}\}}}

        % Ejercicio 3.b
        \item \texttt{\textbf{\{with \{y \{+ x v\}\} \{with \{z x\} 
        \{- x \{- y z\}\}\}\} [x := \{- y z\}]}}

        \textsc{Solución:}
        \texttt{\textbf{\{with \{\textcolor{blue}{y} \{+ \{- \textcolor{green}{y} \textcolor{green}{z}\} 					\textcolor{green}{v}\}\} \{with \{\textcolor{blue}{z} \{- \textcolor{red}{y} \textcolor{green}{z}\}\} 
        \{- \{- \textcolor{red}{y} \textcolor{red}{z}\} \{- \textcolor{red}{y} \textcolor{red}{z}\}\}\}\}}}
        
        % Ejercicio 3.c
        \item \texttt{\textbf{\{with \{y \{- z 3\}\} \{+ x \{+ y 11\}\}\} 
        [x := \{- y \{z 23\}\}]}}

        \textsc{Solución:}
        \texttt{\textbf{\{with \{\textcolor{blue}{y} \{- \textcolor{green}{z} 3\}\} \{+ \{- \textcolor{red}{y} 
        \{\textcolor{green}{z} 23\}\} \{+ \textcolor{red}{y} 11\}\}\}}}
        
    \end{enumerate}

    % Ejercicio 4.
    \item Convierte las siguientes expresiones a su respectiva versión usando
    índices de \textit{De Brujin}.
    \begin{enumerate}
        % Ejercicio 4.a
        \item 
        \begin{verbatim}
        {with {a 2} 
           {with {b 3} 
              {with {c 4} 
                 {with {d {+ a {- b c}}} 
                    {with {f {with {a {+ b c}} a}} 
                       {+ d {with {b {- d f}} {- b c}}}}}}}}
        \end{verbatim}

        \textsc{Solución:}
        \begin{verbatim}
        {with 2 
           {with 3 
              {with 4
                 {with {+ <:2> {- <:1> <:0>}} 
                    {with {with {+ <:2> <:1>} <:0>} 
                       {+ <:1> {with {- <:1> <:0>} {- <:0> <:2>}}}}}}}}
        \end{verbatim}

        % Ejercicio 4.b
        \item 
        \begin{verbatim}
        {with {{a 2} {b 3} {c {with {{a 2}} {+ 2 3}}}} 
           {with {{d 8}} 
              {with {{a c} {b {- 8 d}} {c {+ b b}}} {
                 {with {{g {with {{z a} {y b} {z d}} 1}}} 
                    {+ g {- d c}}}}}}}
        \end{verbatim}

        \textsc{Solución:}
        \begin{verbatim}
        {with {2 3 {with {2} {+ 2 3}}} 
           {with {8} 
              {with {{<:1, 2>} {{- 8 <:0, 0>}} {{+ <:1, 1> <:1, 1>}}} 
                 {with {{with {{<:0, 0>} {<:0, 1>} {<:1, 0>}} 1}} 
                    {+ <:0, 0> {- <:2, 0> <:1, 2>}}}}}}
        \end{verbatim}
    \end{enumerate}

    % Ejercicio 5.
    \item Dadas las siguientes expresiones representadas mediante índices de 
    \textit{De Brujín}, obtén su respectiva versión usando identificadores de 
    variables.
    \begin{enumerate}
        % Ejercicio 5.a
        \item 
        \begin{verbatim}
        {with {+ 2 3} 
           {with 17 
             {with {+ <:0> <:0>} 
                {with {- <:0> {+ <:1> <:2>}} 
                   {with {with 2 {+ <:0> 3}} 
                      {- <:3> {+ <:2> {+ <:0> <:1>}}}}}}}}
        \end{verbatim}

        \textsc{Solución:}
        \begin{verbatim}
        {with {x {+ 2 3}} 
           {with {y 17} 
             {with {z {+ y y}} 
                {with {w {- z {+ y x}}} 
                   {with {v {with {a 2} {+ a 3}}} 
                      {- y {+ z {+ v w}}}}}}}}
        \end{verbatim}

        % Ejercicio 5.b
        \item 
        \begin{verbatim}
        {with {1 2 3} 
           {with {4 5 6} 
              {with {{with {{+ <:0 1> <:1 2>} {- <:1 1> <:0 0>}} 3}} 
              	 {with {<:0 0>}
                 	{+ <:3 2> {+ <:2 1> {+ <:1 0> <:0 0>}}}}}}}
        \end{verbatim}

        \textsc{Solución:}
        \begin{verbatim}
        {with {{a 1} {b 2} {c 3}} 
           {with {{d 4} {e 5} {f 6}} 
              {with {{g {with {{h {+ e c}} {i {- b d}}} 3}}}
              	 {with {j g}
                 	{+ c {+ e {+ g j}}}}}}}
        \end{verbatim}
    \end{enumerate}

    % Ejercicio 6.
    \item Determina el valor de la siguiente expresión y responde las siguientes 
    preguntas: ¿puede haber otro resultado correcto? ¿por qué? ¿cuál es el 
    correcto? 
    \begin{verbatim}
    {with {a 2} 
       {with {b 3} 
          {with {c {+ a b}} 
             {with {a -2} 
                {with {b -3} 
                   {+ c c}}}}}}
    \end{verbatim}

    \textsc{Solución:} La expresion nos puede dar dos diferentes valores dependiendo si se ocupa alcance dinamico 	o estatico y los dos valores son correctos dependiendo de como este implementado el leguaje en el que 				estemos trabajando.
    
    Usando alcance dinamico nos da: -10
    
    Usando alcance estatico nos da:  10
\end{enumerate}

\end{document}
