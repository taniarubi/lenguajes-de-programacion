\documentclass[letterpaper,11pt]{article}

% Soporte para los acentos.
\usepackage[utf8]{inputenc}
\usepackage[T1]{fontenc}    
% Idioma español.
\usepackage[spanish,mexico, es-tabla]{babel}
% Soporte de símbolos adicionales (matemáticas)
\usepackage{multirow}
\usepackage{amsmath}
\usepackage{amssymb}
\usepackage{amsthm}
\usepackage{amsfonts}
\usepackage{mathtools}
\usepackage{latexsym}
\usepackage{enumerate}
\usepackage{ragged2e}
\usepackage{graphicx}
\usepackage{hyperref}
\usepackage{xcolor}
\usepackage{drawstack}
% Modificamos los márgenes del documento.
\usepackage[lmargin=1cm,rmargin=1cm,top=1.5cm,bottom=1.5cm]{geometry}

\tikzstyle{freecell}=[fill=blue!10,draw=blue!30!black]
\tikzstyle{occupiedcell}=[fill=blue!10!orange!10,draw=blue!30!black]
\tikzstyle{padding}=[fill=yellow!20,draw=blue!30!black]
\tikzstyle{highlight}=[draw=orange!50!black,text=orange!50!black]

\title{Facultad de Ciencias, UNAM \\ 
       Lenguajes de Programación \\ 
       Tarea 4}
\author{Rubí Rojas Tania Michelle}
\date{23 de noviembre de 2020}

\begin{document}
\maketitle

\begin{enumerate}
    % Ejercicio 1.
    \item Currifica cada uno de los siguientes términos:
    \begin{enumerate}
        % Ejercicio 1.a
        \item $\lambda abc.abc$

        \textsc{Solución:}
        \begin{equation*}
            \lambda abc.abc \rightarrow 
            \lambda a. \lambda b. \lambda c.abc
        \end{equation*}

        % Ejercicio 1.b
        \item $\lambda abc. \lambda cde.acbdce$
        
        \textsc{Solución:}
        \begin{equation*}
            \lambda abc. \lambda cde.acbdce \rightarrow
            \lambda a. \lambda b. \lambda c. \lambda c. \lambda d. 
            \lambda e.acbdce
        \end{equation*}

        % Ejercicio 1.c
        \item $(\lambda x.(\lambda xy.y) (\lambda zw.w)) (\lambda uv.v)$

        \textsc{Solución:}
        \begin{equation*}
            (\lambda x.(\lambda xy.y) (\lambda zw.w)) (\lambda uv.v) \rightarrow
            (\lambda x. (\lambda x. \lambda y. y) (\lambda z. \lambda w.w))
            (\lambda u. \lambda v.v)
        \end{equation*}
    \end{enumerate}

    % Ejercicio 2.
    \item Para cada uno de los siguientes términos, aplica $\alpha-$conversiones
    para obtener términos donde todas las variables de ligado sean distintas.
    \begin{enumerate}
        % Ejercicio 2.a
        \item $\lambda x. \lambda y. \; (\lambda x.y \; \lambda y.x)$

        \textsc{Solución:}
        \begin{align*}
            \lambda x. \lambda y. \; (\lambda x.y \; \lambda y.x)
            &\equiv_{\alpha}
            \lambda a. \lambda y. \;(\lambda x.y \; \lambda y.x) [x := a] \\
            &\equiv_{\alpha} 
            \lambda a. \lambda y \; (\lambda x.y \; \lambda y.a) \\
            &\equiv_{\alpha}
            \lambda a. \lambda b. \; (\lambda x.y \; \lambda y.a) [y := b] \\
            &\equiv_{\alpha}
            \lambda a. \lambda b. \; (\lambda x.b \; \lambda y.a) 
        \end{align*}

        % Ejercicio 2.b
        \item $\lambda x. \; (x \; (\lambda y. \; (\lambda x.x \; y) \; x))$
        
        \textsc{Solución:}
        \begin{align*}
            \lambda x. \; (x \; (\lambda y. \; (\lambda x.x \; y) \; x))
            &\equiv_{\alpha}
            \lambda a. \; (x \; (\lambda y. \; (\lambda x.x \; y) \; x)) 
            [x := a] \\
            &\equiv_{\alpha} 
            \lambda a. \; (a \; (\lambda y. \; (\lambda x.x \; y) \; a))
        \end{align*}

        % Ejercicio 2.c
        \item $\lambda a. \; (\lambda b.a \; \lambda b \; (\lambda a.a \; b))$

        \textsc{Solución:}
        \begin{align*}
            \lambda a. \; (\lambda b.a \; \lambda b \; (\lambda a.a \; b))
            &\equiv_{\alpha}
            \lambda x. \; (\lambda b.a \; \lambda b \; (\lambda a.a \; b))
            [a := x] \\
            &\equiv_{\alpha}
            \lambda x. \; (\lambda b.x \; \lambda b \; (\lambda a.a \; b)) \\
            &\equiv_{\alpha}
            \lambda x. \; (\lambda y.x[b := y] \; \lambda z[b := z] \; 
            (\lambda a.a \; b)) \\
            &\equiv_{\alpha}
            \lambda x. \; (\lambda y.x \; \lambda z \; (\lambda a.a \; b)) \\
        \end{align*}
    \end{enumerate}

    % Ejercicio 3.
    \item Aplicar las $\beta-$reducciones correspondientes a las siguientes 
    expresiones hasta llegar a una Forma Normal o justificar por qué dicha 
    forma no existe. Indicar en cada paso la \textit{redex} y el 
    \textit{reducto}. Considerar las siguientes definiciones:
    \begin{equation*}
        I =_{\text{def}} \lambda x.x \quad \quad 
        S =_{\text{def}} \lambda x. \lambda y. \lambda z.xz (yz)
    \end{equation*}
    \begin{equation*}
        K =_{\text{def}} \lambda x. \lambda y.x \quad \quad 
        \Omega =_{\text{def}} (\lambda x.xx)(\lambda x.xx)
    \end{equation*}

    \begin{enumerate}
        % Ejercicio 3.a
        \item $\lambda x.xK\Omega$

        \textsc{Solución:} Como la expresión ya no puede reducirse más mediante 
        $\beta-$reducciones ya que no tiene espacios (y por lo tanto, no tiene 
        aplicaciones de función), entonces ya se encuentra en Forma Normal.

        % Ejercicio 3.b
        \item $(\lambda x.x \; (II)) \; z$

        \textsc{Solución:} Tenemos que 
        \begin{align*}
            (\lambda x.x \; (II)) \; z 
            &=_{def} (\lambda x.x \; (\textcolor{blue}{\lambda x.x} \;
                                      \lambda x.x)) \; z \\
            &\rightarrow_{\beta} (\lambda x.x \; 
                                 (\textcolor{red}{x [x := \lambda x.x]})) 
                                 \; z \\
            &\rightarrow_{\beta}
            (\textcolor{blue}{\lambda x.x} \; (\lambda x.x)) \; z \\ 
            &\rightarrow_{\beta} (\textcolor{red}{x [x := (\lambda x.x)]}) \; z \\ 
            &\rightarrow_{\beta} \textcolor{blue}{(\lambda x.x)} \; z \\
            &\rightarrow_{\beta} \textcolor{red}{x [x := z]} \\
            &\rightarrow_{\beta} z
        \end{align*}

        donde las expresiones en color \textcolor{blue}{azul} son el 
        \textit{redex} y las expresiones en color \textcolor{red}{rojo} son el 
        \textit{reducto}. Ahora bien, como la expresión \texttt{z} ya no puede
        reducirse más mediante $\beta-$ reducciones, entonces ya se encuentra 
        en Forma Normal.

        % Ejercicio 3.c
        \item $(\lambda u. \lambda v. \; (\lambda w.w \; (\lambda x.xu)) \; v) 
        \; y \; (\lambda z. \lambda y.zy)$

        \textsc{Solución:} Tenemos que 
        \begin{align*}
            (\textcolor{blue}{\lambda u. \lambda v. \; (\lambda w.w \; 
                                                       (\lambda x.xu))} \; v) 
            \; y \; (\lambda z. \lambda y.zy)
            &\rightarrow_{\beta}
            (\textcolor{red}{\lambda v. \; (\lambda w.w \; (\lambda x.xu))
                             [u := v]}) \; y \; (\lambda z. \lambda y.zy) \\
            &\rightarrow_{\beta}
            (\lambda v. \; \textcolor{red}{(\lambda w.w \; (\lambda x.xu))
            [u := v]}) \; y \; (\lambda z. \lambda y.zy) \\
            &\rightarrow_{\beta} 
            (\lambda v. \; \textcolor{red}{(\lambda w.w [u := v] \; 
            (\lambda x.xu) [u := v])}) \; y \; (\lambda z. \lambda y.zy) \\
            &\rightarrow_{\beta} 
            (\lambda v. \; (\lambda w. \textcolor{red}{w [u := v]} \;
            (\lambda x. \textcolor{red}{xu [u := v]}))) \; y \; 
            (\lambda z. \lambda y.zy) \\
            &\rightarrow_{\beta}
            (\lambda v. \; (\lambda w.w \; (\lambda x. 
            \textcolor{red}{x[u := v] \; u[u := v]}))) \; y \; 
            (\lambda z. \lambda y.zy) \\
            &\rightarrow_{\beta}
            \textcolor{blue}{(\lambda v. \; (\lambda w.w \; (\lambda x.xv)))}
            \; y \; (\lambda z. \lambda y.zy) \\
            &\rightarrow_{\beta}
            \textcolor{red}{(\lambda w.w \; (\lambda x.xv)) [v := y]}
            \; (\lambda z. \lambda y.zy) \\
            &\rightarrow_{\beta}
            \textcolor{red}{(\lambda w.w [v := y] \; (\lambda x.xv) [v := y])}
            \; (\lambda z. \lambda y.zy) \\
            &\rightarrow_{\beta}
            (\lambda w. \textcolor{red}{w [v := y]} \; (\lambda x.
            \textcolor{red}{xv [v := y]})) \; (\lambda z. \lambda y.zy) \\
            &\rightarrow_{\beta}
            (\lambda w.w \; (\lambda x. 
            \textcolor{red}{x [v := y] \; v [v := y]}))
            \; (\lambda z. \lambda y.zy) \\
            &\rightarrow_{\beta}
            (\textcolor{blue}{\lambda w.w} \; (\lambda x.xy))
            \; (\lambda z. \lambda y.zy) \\
            &\rightarrow_{\beta}  
            (\textcolor{red}{w [w := (\lambda x.xy)]}) \; 
            (\lambda z. \lambda y.zy)\\
            &\rightarrow_{\beta}  
            \textcolor{blue}{(\lambda x.xy)} \; 
            (\lambda z. \lambda y.zy) \\
            &\rightarrow_{\beta}  
            \textcolor{red}{xy [x := (\lambda z. \lambda y.zy)]} \\
            &\rightarrow_{\beta}
            \textcolor{red}{x [x := (\lambda z. \lambda y.zy)] \; 
                            y [x := (\lambda z. \lambda y.zy)]} \\
            &\rightarrow_{\beta}
            (\lambda z. \lambda y.\textcolor{blue}{zy}) y \\ 
            &\rightarrow_{\beta}
            \textcolor{red}{\lambda y.zy [z := y]} \\
            &\rightarrow_{\beta}
            \lambda y. \textcolor{red}{zy [z := y]} \\
            &\rightarrow_{\beta}
            \lambda y. \textcolor{red}{z [z := y] \; 
                                        y [z := y]} \\
            &\rightarrow_{\beta} \lambda y.yy
        \end{align*}

        donde las expresiones en color \textcolor{blue}{azul} son el 
        \textit{redex} y las expresiones en color \textcolor{red}{rojo} son el 
        \textit{reducto}. Ahora bien, como la expresión $\lambda y.yy$ ya no 
        puede reducirse más mediante $\beta-$ reducciones, entonces ya se 
        encuentra en Forma Normal.

        % Ejercicio 3.d
        \item $S \; (KI) \; (KI)$

        \textsc{Solución:} Tenemos que 
        \begin{align*}
            S \; (KI) \; (KI) 
            &\equiv (S \; (KI)) \; (KI) \\ 
            &=_{def} (S \; (\textcolor{blue}{\lambda x. \lambda y.x} \; 
                            \lambda x.x)) \;
                           (\textcolor{blue}{\lambda x. \lambda y.x} \; 
                            \lambda x.x) \\
            &\rightarrow_{\beta}
            (S \; (\textcolor{red}{\lambda y.x [x := \lambda x.x]})) \;
            (\textcolor{red}{\lambda y.x [x := \lambda x.x]}) \\
            &\rightarrow_{\beta}
            (S \; (\lambda y. \textcolor{red}{x [x := \lambda x.x]})) \; 
            (\lambda y. \textcolor{red}{x [x := \lambda x.x]}) \\
            &\rightarrow_{\beta} 
            (S \; (\lambda y. \lambda x.x)) \; (\lambda y. \lambda x.x) \\
            &=_{def} (\textcolor{blue}{\lambda x. \lambda y. \lambda z.xz} \; 
                     (yz) \; (\lambda y. \lambda x.x)) \; 
                     (\lambda y. \lambda x.x) \\
            &\rightarrow_{\beta}
            (\textcolor{red}{\lambda y. \lambda z.xz [x := (yz)]} \; 
            (\lambda y. \lambda x.x)) \; (\lambda y. \lambda x.x) \\
            &\rightarrow_{\beta}
            (\lambda y. \textcolor{red}{\lambda z.xz [x := (yz)]} \; 
            (\lambda y. \lambda x.x) \; (\lambda y. \lambda x.x) \\ 
            &\rightarrow_{\beta}
            (\lambda y. \lambda z. \textcolor{red}{xz [x := (yz)]} \; 
            (\lambda y. \lambda x.x)) \; (\lambda y. \lambda x.x) \\
            &\rightarrow_{\beta}
            (\lambda y. \lambda z. \textcolor{red}{x [x := (yz)] \; 
            z [x := (yz)]} 
            \; (\lambda y. \lambda x.x)) \; (\lambda y. \lambda x.x) \\
            &\rightarrow_{\beta}
            (\textcolor{blue}{\lambda y. \lambda z.(yz)z} \; 
            (\lambda y. \lambda x.x)) \; (\lambda y. \lambda x.x) \\
            &\rightarrow_{\beta}
            (\textcolor{red}{\lambda z.(yz)z 
            [y := (\lambda y. \lambda x.x)]}) \; (\lambda y. \lambda x.x) \\
            &\rightarrow_{\beta}
            (\lambda z. \textcolor{red}{(yz)z [y := (\lambda y. \lambda x.x)]}) 
            \; (\lambda y. \lambda x.x) \\
            &\rightarrow_{\beta}
            (\lambda z. \textcolor{red}{(yz) [y := (\lambda y. \lambda x.x)] \; 
                                         z[y := (\lambda y. \lambda x.x)]}) 
            \; (\lambda y. \lambda x.x) \\
            &\rightarrow_{\beta} 
            (\lambda z. \textcolor{red}{(y [y := (\lambda y. \lambda x.x)] \; 
                                          z [y := (\lambda y. \lambda x.x)])} 
            \; z) \; 
            (\lambda y. \lambda x.x) \\ 
            &\rightarrow_{\beta}
            (\lambda z. (\textcolor{blue}{(\lambda y. \lambda x.x)} z)z) \; 
            (\lambda y. \lambda x.x) \\
            &\rightarrow_{\beta}
            (\lambda z. (\textcolor{red}{\lambda x.x [y := z]}) z) \; 
            (\lambda y. \lambda x.x) \\
            &\rightarrow_{\beta}
            (\lambda z. (\lambda x. \textcolor{red}{x [y := z]}) z) \; 
            (\lambda y. \lambda x.x) \\
            &\rightarrow_{\beta}
            (\lambda z. \textcolor{blue}{(\lambda x. x)} z) \; 
            (\lambda y. \lambda x.x) \\
            &\rightarrow_{\beta}
            (\lambda z. (\textcolor{red}{x [x =: z]})) \; 
            (\lambda y. \lambda x.x) \\
            &\rightarrow_{\beta}
            \textcolor{blue}{(\lambda z. z)} \; 
            (\lambda y. \lambda x.x) \\
            &\rightarrow_{\beta}
            \textcolor{red}{z [z := (\lambda y. \lambda x.x)]} \\
            &\rightarrow_{\beta} (\lambda y. \lambda x.x)
        \end{align*}

        donde las expresiones en color \textcolor{blue}{azul} son el 
        \textit{redex} y las expresiones en color \textcolor{red}{rojo} son el 
        \textit{reducto}. Ahora bien, como la expresión $\lambda y. \lambda x.x$
        ya no puede reducirse más mediante $\beta-$reducciones, entonces ya se 
        encuentra en Forma Normal.
    \end{enumerate}

    % Ejercicio 4.
    \item De acuerdo a la representación de números (Numerales de Church) y 
    representación de booleanos en el Cálculo $\lambda$:
    \begin{enumerate}
        % Ejercicio 4.a
        \item Define la función $<$ que decide si un número es menor a otro.

        \textsc{Solución:} Como ya tenemos definida la función $\geq$, entonces 
        podemos definir la función $<$ como la negación de $\geq$; pues si 
        $x$ no es estrictamente mayor o igual a $y$ entonces eso implica que 
        necesariamente $x$ es menor que $y$.
        Por lo tanto, 
        \begin{equation*}
            < \; =_{def} \; (\lambda a. \lambda b. \neg \; (\geq ab))
        \end{equation*}

        % Ejercicio 4.b.
        \item Define la función disyunción $\leftrightarrow$ (equivalencia) 
        sobre booleanos.

        \textsc{Solución:} Sabemos que
        \begin{equation*}
            p \rightarrow q \equiv \neg p \lor q \quad \quad \quad \quad 
            p \leftrightarrow q \equiv (p \rightarrow q) \land (q \rightarrow p)
        \end{equation*}

        Por lo que podemos definir las funciones 
        \begin{equation*}
            \rightarrow \; =_{def} \; (\lambda a. \lambda b. \lor (\neg a) \; b)
            \quad \quad \quad 
            \leftrightarrow \; =_{def} \; 
            (\lambda a. \lambda b. \land (\rightarrow ab) \; (\rightarrow ba))
        \end{equation*}

        % Ejercicio 4.c
        \item Define la función disyunción exclusiva \textit{xor} sobre 
        booleanos.

        \textsc{Solución:}
        \begin{equation*}
            xor =_{def} (\lambda a. \lambda b.a \; (bFT) \; (bTF))
        \end{equation*}
    \end{enumerate}

    % Ejercicio 5.
    \item Dada la siguiente expresión en \textsc{racket}:
    \begin{equation*}
        \texttt{(let ([sum (} \lambda \; 
        \texttt{(n) (if (zero? n) 0 (+ n (sum (sub1 n)))))])} \; 
        \texttt{(sum 5))}
    \end{equation*}

    \begin{enumerate}
        % Ejercicio 5.a
        \item Ejecútala y explica el resultado.
        
        \textsc{Solución:} Nos regresa un error del tipo 
        \textcolor{red}{sum: unbound identifier in: sum}, lo que significa 
        que hubo un error de variable libre. Esto sucede porque al momento 
        de querer pasarle una llamada a la función, entonces \textsc{Racket}
        no sabe que esa función es \texttt{sum} (pues la función es anónima), 
        así que al momento de querer buscar la definición de \texttt{sum} no 
        la encuentra; por lo que nos genera un error de variable libre.

        % Ejercicio 5.b
        \item Modifícala usando el Combinador de Punto Fijo Y. Ejecútala y 
        explica el resultado.

        \textsc{Solución:} Para usar el combinador Y debemos adaptar nuestra
        definición de tal forma que nuestra función reciba una función como 
        parámetro, es decir, 
        \begin{equation*}
            \texttt{(let ([sum (} \lambda \texttt{(sum) \; (} \lambda 
            \texttt{(n) (if (zero? n) 0 (+ n (sum (sub1 n)))))}
            \texttt{)]) (sum 5))}
        \end{equation*}

        De esta forma, podemos definir $Y$ mediante un identificador y 
        aplicarlo a \texttt{sum}; esto es,
        \begin{equation*}
            \texttt{(let* ([Y (lambda (f) ((lambda(x) (f (x x))) 
            (lambda (x) (f (x x)))))]}
        \end{equation*}
        \begin{equation*}
            \texttt{[sum (Y (lambda(sum) (lambda (n) (if (zero? n) 0 
            (+ n (sum (sub1 n)))))))]) (sum 5))}
        \end{equation*}

        Finalmente, al ejecutar esta expresión obtenemos un error del tipo 
        \textcolor{red}{Interactions disabled; out of memory}, lo que 
        significa que el programa se cicló. Esto pasa por la definición del 
        combinador $Y$ y porque \textsc{Racket} tiene un régimen de 
        evaluación glotón. Como en cada llamada se aplica el combinador $Y$, 
        entonces en cada pasito se genera una nueva aplicación de la función 
        $\lambda$\texttt{(sum)}, lo que genera llamadas infinitas; esto 
        ocurre pues llamada tras llamada el parámetro real de la función se 
        evalúa aunque no sea necesario (ya que es glotón).
    \end{enumerate}
\end{enumerate}

\end{document}
