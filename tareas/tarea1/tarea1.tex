\documentclass[letterpaper,11pt]{article}

% Soporte para los acentos.
\usepackage[utf8]{inputenc}
\usepackage[T1]{fontenc}    
% Idioma español.
\usepackage[spanish,mexico, es-tabla]{babel}
% Soporte de símbolos adicionales (matemáticas)
\usepackage{multirow}
\usepackage{amsmath}		
\usepackage{amssymb}		
\usepackage{amsthm}
\usepackage{amsfonts}
\usepackage{latexsym}
\usepackage{enumerate}
\usepackage{ragged2e}
\usepackage{hyperref}
% Modificamos los márgenes del documento.
\usepackage[lmargin=2cm,rmargin=2cm,top=2cm,bottom=2cm]{geometry}

\title{Facultad de Ciencias, UNAM \\ 
       Lenguajes de Programación \\ 
       Tarea 1}
\author{ \\ 
        Rubí Rojas Tania Michelle }
\date{09 de octubre de 2020}

\begin{document}
\maketitle

\begin{enumerate}
    % Ejercicio 1.
    \item Explica brevemente qué tipos de problemas puedes resolver con cada uno
    de los siguientes paradigmas y nombre un lenguaje perteneciente a cada uno.
    \begin{enumerate}
        % Ejercicio 1.a
        \item Paradigma Estructurado.
        % Ejercicio 1.b
        \item Paradigma Orientado a Objetos.
        % Ejercicio 1.c
        \item Paradigma Funcional.
        % Ejercicio 1.d
        \item Paradigma Lógico.
    \end{enumerate}

    % Ejercicio 2.
    \item Usando el lenguaje de programación \textsc{Prolog}, dar un ejemplo de 
    cada uno de los siguientes conceptos y justificar. No es necesario que sean 
    ejemplos demasiado elaborados. 
    \begin{enumerate}
        % Ejercicio 2.a
        \item Sintaxis
        % Ejercicio 2.b
        \item Semántica
        % Ejercicio 2.c
        \item Convenciones de programación (Idioms)
        % Ejercicio 2.d
        \item Bibliotecas
    \end{enumerate}

    % Ejercicio 3.
    \item Dada la siguiente función, da una forma para la misma indicando el 
    tipo de entrada de los parámetros, el tipo de la salida y asígnale un nombre 
    mnemotécnico. Justifica tu respuesta. 
    \begin{verbatim}
        (define (foo n 1)
           (cond 
              [(zero? n) empty]
              [else (cons (car l) (foo (sub1 n) (cdr 1)))]))
    \end{verbatim}

    % Ejercicio 4.
    \item Para los siguientes incisos, calcular el resultado de aplicar la 
    función al parámetro recibido. Mostrar cada paso realizado hasta obtener el 
    resultado final. Da tu propia implementación para ambas funciones.
    \begin{enumerate}
        % Ejercicio 4.a
        \item \texttt{(reverse '(1 7 2 9))}
        % Ejercicio 4.b
        \item \texttt{(append '(m a n) (z a n a))}
        % Ejercicio 4.c
        \item \texttt{(reverse (append '(m a n) (z a n a)))}
    \end{enumerate} 

    % Ejercicio 5.
    \item Da una tabla donde expliques las principales diferencias entre un 
    compilador y un intérprete.

    % Ejercicio 6.
    \item Dibuja un mapa mental que muestre las fases de generación de código
    ejecutable, sus principales características y elementos involucrados.

    % Ejercicio 7.
    \item Dadas las siguientes expresiones de AE en sintaxis concreta, da su 
    respectiva representación en sintaxis abstracta por medio de los Árboles 
    de Sintaxos Abstracta correspondientes. En caso de no poder generar el
    árbol, justificar. 
\end{enumerate}

\end{document}
