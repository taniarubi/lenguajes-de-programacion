\documentclass[letterpaper,11pt]{article}

% Soporte para los acentos.
\usepackage[utf8]{inputenc}
\usepackage[T1]{fontenc}
% Idioma español.
\usepackage[spanish,mexico, es-tabla]{babel}
% Soporte de símbolos adicionales (matemáticas)
\usepackage{amsmath}
\usepackage{amssymb}
\usepackage{amsthm}
\usepackage{amsfonts}
\usepackage{mathtools}
\usepackage{latexsym}
\usepackage{enumerate}
\usepackage{ragged2e}
\usepackage[dvipsnames]{xcolor}
\usepackage{drawstack}
\usepackage{float}
% Modificamos los márgenes del documento.                                       
\usepackage[lmargin=1.5cm,rmargin=1.5cm,top=1.5cm,bottom=1.5cm]{geometry}

\tikzstyle{freecell}=[fill=blue!10,draw=blue!30!black]
\tikzstyle{occupiedcell}=[fill=blue!10!orange!10,draw=blue!30!black]
\tikzstyle{padding}=[fill=yellow!20,draw=blue!30!black]
\tikzstyle{highlight}=[draw=orange!50!black,text=orange!50!black]

\title{Facultad de Ciencias, UNAM \\ 
       Lenguajes de Programación\\ 
       Tarea 5}
\author{Rubí Rojas Tania Michelle}
\date{07 de diciembre de 2020}

\begin{document}
\maketitle

\begin{enumerate}
    % Ejercicio 1.
    \item Evalúa la siguiente expresión usando el tipo de alcance y régimen de 
    evaluación que se indica. Es necesario incluir el ambiente final en forma 
    de pila en cada caso. 
    \begin{verbatim}
    {with {a {+ 2 2}}
       {with {b {+ a a}}
          {with {foo {fun {x} {- x b}}
             {with {a {- 2 2}}
                {with {b {- a a}}
                   {foo -3}}}}}}}
    \end{verbatim}

    \begin{enumerate}
        % Ejercicio 1.a
        \item Alcance estático y evaluación glotona

        \textsc{Solución:} La expresión que debemos evaluar es 
        \texttt{\{foo -3\}}, por lo que 
        \begin{table}[h]
            \parbox{.6\linewidth}{
            \centering
            \begin{align*}
                \texttt{\{foo -3\}} 
                &= \texttt{\{\{fun \{x\} \{- x b\}\} -3\}} \\
                &= \texttt{\{- (-3) b\}} \\
                &= \texttt{\{- (-3) 8\}} \\
                &= -11
            \end{align*}
            }
            \hfill
            \parbox{.4\linewidth}{
            \centering
            \begin{drawstack}[scale=1.15]
                \cell{\texttt{b} \quad \quad $0$} 
                \cell{\texttt{a} \quad \quad $0$}
                \cell{\texttt{foo} \quad 
                      \texttt{\{fun \{x\} \{- x b\}\}}}
                \bcell{\texttt{x} \quad \quad $-3$}
                \cell{\texttt{b} \quad \quad $8$}
                \cell{\texttt{a} \quad \quad $4$}
            \end{drawstack}
            }
        \end{table}

        \newpage
        % Ejercicio 1.b
        \item Alcance dinámico y evaluación glotona 

        \textsc{Solución:} La expresión que debemos evaluar es 
        \texttt{\{foo -3\}}, por lo que 
        \begin{table}[h]
            \parbox{.6\linewidth}{
            \centering
            \begin{align*}
                \texttt{\{foo -3\}}
                &= \texttt{\{\{fun \{x\} \{- x b\}\} -3\}} \\
                &= \texttt{\{- (-3) 0\}} \\ 
                &= \texttt{-3}
            \end{align*}
            }
            \hfill
            \parbox{.4\linewidth}{
            \centering
            \begin{drawstack}[scale=1.15]
                \bcell{\texttt{x} \quad \quad $-3$}
                \cell{\texttt{b} \quad \quad $0$} 
                \cell{\texttt{a} \quad \quad $0$}
                \cell{\texttt{foo} \quad 
                      \texttt{\{fun \{x\} \{- x b\}\}}}
                \cell{\texttt{b} \quad \quad $8$}
                \cell{\texttt{a} \quad \quad $4$}
            \end{drawstack}
            }
        \end{table}

        % Ejercicio 1.c
        \item Alcance estático y evaluación perezosa

        \textsc{Solución:} La expresión que debemos evaluar es 
        \texttt{\{foo -3\}}, por lo que 
        \begin{table}[h]
            \parbox{.6\linewidth}{
            \centering
            \begin{align*}
                \texttt{\{foo -3\}}
                &= \texttt{\{\{fun \{x\} \{- x b\}\} -3\}} \\
                &= \texttt{\{- (-3) b\}} \\
                &= \texttt{\{- (-3) \{+ a a\}\}} \\
                &= \texttt{\{- (-3) \{+ \{+ 2 2\} \{+ 2 2\}\}\}} \\
                &= \texttt{\{- (-3) \{+ 4 4\}\}} \\
                &= \texttt{\{- (-3) 8\}} \\
                &= \texttt{-11}
            \end{align*}
            }
            \hfill
            \parbox{.4\linewidth}{
            \centering
            \begin{drawstack}[scale=1.15]
                \cell{\texttt{b} \quad \quad
                      \texttt{\{- a a\}}} 
                \cell{\texttt{a} \quad \quad 
                      \texttt{\{- 2 2\}}}
                \cell{\texttt{foo} \quad 
                      \texttt{\{fun \{x\} \{- x b\}\}}}
                \bcell{\texttt{x} \quad \quad $-3$}
                \cell{\texttt{b} \quad \quad 
                      \texttt{\{+ a a\}}}
                \cell{\texttt{a} \quad \quad 
                      \texttt{\{+ 2 2\}}}
            \end{drawstack}
            }
        \end{table}

        \newpage
        % Ejercicio 1.d
        \item Alcance dinámico y evaluación perezosa

        \textsc{Solución:} La expresión que debemos evaluar es 
        \texttt{\{foo -3\}}, por lo que 
        \begin{table}[h]
            \parbox{.6\linewidth}{
            \centering
            \begin{align*}
                \texttt{\{foo -3\}}
                &= \texttt{\{\{fun \{x\} \{- x b\}\} -3\}} \\
                &= \texttt{\{- (-3) b\}} \\
                &= \texttt{\{- (-3) \{- a a\}\}} \\
                &= \texttt{\{- (-3) \{- \{- 2 2\} \{- 2 2\}\}\}} \\
                &= \texttt{\{- (-3) \{- 0 0\}\}} \\
                &= \texttt{\{- (-3) 0\}} \\
                &= \texttt{-3}
            \end{align*}
            }
            \hfill
            \parbox{.4\linewidth}{
            \centering
            \begin{drawstack}[scale=1.15]
                \bcell{\texttt{x} \quad \quad $-3$}
                \cell{\texttt{b} \quad \quad
                      \texttt{\{- a a\}}} 
                \cell{\texttt{a} \quad \quad 
                      \texttt{\{- 2 2\}}}
                \cell{\texttt{foo} \quad 
                      \texttt{\{fun \{x\} \{- x b\}\}}}
                \cell{\texttt{b} \quad \quad 
                      \texttt{\{+ a a\}}}
                \cell{\texttt{a} \quad \quad 
                      \texttt{\{+ 2 2\}}}
            \end{drawstack}
            }
        \end{table}
    \end{enumerate}
\end{enumerate}

\end{document}
